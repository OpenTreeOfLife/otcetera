%% LyX 2.1.4 created this file.  For more info, see http://www.lyx.org/.
%% Do not edit unless you really know what you are doing.
\documentclass[english]{article}
\usepackage[T1]{fontenc}
\usepackage[latin9]{inputenc}
\usepackage{geometry}
\geometry{verbose,tmargin=2cm,bmargin=2cm,lmargin=2cm,rmargin=2cm}
\usepackage{babel}
\begin{document}

\title{Thoughts on constructing the Full Supertree}

\maketitle
In the \emph{otcetera} pipeline, the synthesis tree is first constructed
using a pruned taxonomy. This taxonomy has been pruned to remove any
leaves that do not occur in one of the ranked input phylogenies. Therefore,
a later step in the pipeline involves adding the pruned taxa back
in to the synthesis tree. The resulting tree is the \emph{full supertree.
}This document is an attempt to collect, and perhaps organize, thoughts
on how the full supertree can or should be constructed, as well as
related questions and concepts.

This document takes the approach that we should be able to use the
sub-problem solver to construct the full supertree by feeding it a
sequence of 2 trees:
\begin{verbatim}
otc-solve-subproblem grafted-solution.tre cleaned_ott.tre
\end{verbatim}
Various other quick-and-dirty methods may be sufficient to achieve
the result. However, the attempt to make the subproblem solver fast
enough to solve this particular problem raises various issues with
the subproblem solver. It also suggests various optimizations that
may be useful more generally.


\section{Potential speed increases}

In the current implementation of the BUILD algorithm, we have a number
of places we take excessive computation time:
\begin{itemize}
\item We attempt to construct rooted splits (a.k.a. desIds) for each node.
For a bifurcating tree, this operation should take time and memory
quadratic in the number of leaves.
\item We attempt to construct connected component by considering each split
in a tree separately. However, considering splits for nodes that are
not direct children of the root is redundant.
\item Much of the time is spent in determining which splits are imposed
at a given level in the tree, and therefore need not be passed to
subproblems.

\begin{itemize}
\item When different splits have the same leaf set, we should be able to
get a speedup.
\item When some splits have the full leaf set, we should be able to get
a speedup.
\end{itemize}
\item We recompute connected components from scratch each time BUILD is
recursively called on a subproblem. This could be avoided by incrementally
removing edges from the graph and discovering new connected components
that appear, as in Henzinger et al. \emph{However, it is unclear if
an algorithm similar to Henzinger et al could be used to find the
edges to remove at each step.}
\item When we have two trees $T_{1}$ and $T_{2}$, and either $\mathcal{L}(T_{1})\subseteq\mathcal{L}(T_{2})$
or $\mathcal{L}(T_{2})\subseteq\mathcal{L}(T_{1})$, then it should
be possible to determine all conflicting splits in a single pass over
the trees, similar to \texttt{otc-detectcontested}.
\end{itemize}

\section{The problem}

When the subproblem to be solved consists of two ranked trees, $T_{1}$
and $T_{2}$, and the second tree is the taxonomy, then we have $\mathcal{L}(T_{1})\subseteq\mathcal{L}(T_{2})$.
For each (rooted) split in $T_{2}$, we can determine whether that
split is consistent with $T_{1}$. We claim that each split in $T_{2}$
is either consistent with $T_{1}$, or incompatible with at least
one split of $T_{1}$. We can therefore remove each split of $T_{2}$
that is inconsistent with $T_{1}$ to form a new tree $T_{2}^{\prime}$.
The splits of $T_{1}$ and $T_{2}^{\prime}$ are then jointly consistent.
Furthermore, by combining the trees $T_{1}$ and $T_{2}^{\prime}$,
we obtain a new tree in which the splits of $T_{1}$ that are not
implied by $T_{2}^{\prime}$ may not fully specify where certain taxa
of $T_{2}^{\prime}$ are placed. We resolve this ambiguity by placing
such taxa rootward, but their range of attachment extends over specific
branches in the tree obtained by combining $T_{1}$ and $T_{2}^{\prime}$.
All of these branches derive from $T_{1}$ but not from $T_{2}^{\prime}$.


\section{Questions}
\begin{itemize}
\item Is there a single, unique solution to this problem?
\item When running BUILD, is it possible to construct a \emph{reason} for
the lack of inclusion of each split? Specifically, can we say which
split (or set of splits) conflicts with that split?\end{itemize}

\end{document}
