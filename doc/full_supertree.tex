%% LyX 2.1.4 created this file.  For more info, see http://www.lyx.org/.
%% Do not edit unless you really know what you are doing.
\documentclass[english]{article}
\usepackage[T1]{fontenc}
\usepackage[latin9]{inputenc}
\usepackage{geometry}
\geometry{verbose,tmargin=2cm,bmargin=2cm,lmargin=2cm,rmargin=2cm}
\usepackage{babel}
\begin{document}

\title{Thoughts on constructing the Full Supertree}

\maketitle
In the \emph{otcetera} pipeline, the synthesis tree is first constructed
using a pruned taxonomy. This taxonomy has been pruned to remove any
leaves that do not occur in one of the ranked input phylogenies. Therefore,
a later step in the pipeline involves adding the pruned taxa back
in to the synthesis tree. The resulting tree is the \emph{full supertree.
}This document is an attempt to collect, and perhaps organize, thoughts
on how the full supertree can or should be constructed, as well as
related questions and concepts.

This document takes the approach that we should be able to use the
sub-problem solver to construct the full supertree by feeding it a
sequence of 2 trees:
\begin{verbatim}
otc-solve-subproblem grafted-solution.tre cleaned_ott.tre
\end{verbatim}
\selectlanguage{english}%
Various other quick-and-dirty methods may be sufficient to achieve
the result. However, the attempt to make the subproblem solver fast
enough to solve this particular problem raises various issues with
the subproblem solver. It also suggests various optimizations that
may be useful more generally.


\section{Potential speed increases}

In the current implementation of the BUILD algorithm, we have a number
of places we take excessive computation time:
\begin{itemize}
\item We attempt to construct rooted splits (a.k.a. desIds) for each node.
For a bifurcating tree, this operation should take time and memory
quadratic in the number of leaves.
\item We attempt to construct connected component by considering each split
in a tree separately. However, considering splits for nodes that are
not direct children of the root is redundant.
\item Much of the time is spent in determining which splits are imposed
at a given level in the tree, and therefore need not be passed to
subproblems.

\begin{itemize}
\item When different splits have the same leaf set, we should be able to
get a speedup.
\item When some splits have the full leaf set, we should be able to get
a speedup.
\end{itemize}
\item We recompute connected components from scratch each time BUILD is
recursively called on a subproblem. This could be avoided by incrementally
removing edges from the graph and discovering new connected components
that appear, as in Henzinger et al. \emph{However, it is unclear if
an algorithm similar to Henzinger et al could be used to find the
edges to remove at each step.}
\item When we have two trees $T_{1}$ and $T_{2}$, and either $\mathcal{L}(T_{1})\subseteq\mathcal{L}(T_{2})$
or $\mathcal{L}(T_{2})\subseteq\mathcal{L}(T_{1})$, then it should
be possible to determine all conflicting splits in a single pass over
the trees, similar to \texttt{otc-detectcontested}.
\end{itemize}

\section{The problem}

When the subproblem to be solved consists of two ranked trees, $T_{1}$
and $T_{2}$, and the second tree is the taxonomy, then we have $\mathcal{L}(T_{1})\subseteq\mathcal{L}(T_{2})$.
For each (rooted) split in $T_{2}$, we can determine whether that
split is consistent with $T_{1}$. We claim that each split in $T_{2}$
is either consistent with $T_{1}$, or incompatible with at least
one split of $T_{1}$. We can therefore form a new tree $T_{2}^{\prime}$
by starting with $T_{2}$ and removing each split that is inconsistent
with $T_{1}$. The splits of $T_{1}$ and $T_{2}^{\prime}$ are then
jointly consistent. Furthermore, by combining the trees $T_{1}$ and
$T_{2}^{\prime}$, we obtain a new tree in which the splits of $T_{1}$
that are not implied by $T_{2}^{\prime}$ may not fully specify where
certain taxa of $T_{2}^{\prime}$ are placed. We resolve this ambiguity
by placing such taxa rootward, but their range of attachment extends
over specific branches in the tree obtained by combining $T_{1}$
and $T_{2}^{\prime}$. All of these branches derive from $T_{1}$
but not from $T_{2}^{\prime}$.

First, note that each split of $T_{2}$ implies a split on the reduced
leaf set $\mathcal{L}(T_{1})$. This split then is either consistent
with $T_{1}$, or conflicts with at least one split in $T_{1}$. We
obtain $T_{2}^{\prime}$ by removing each split of $T_{2}$ that,
when reducted to leaf set $\mathcal{L}(T_{1})$, conflicts with some
branch of $T_{1}.$ By definition, the remaining branches of $T_{2}$
are individually compatible with each branch of $T_{1}$.

Second, let us look at the subtree of $T_{2}$ obtained by restricting
the leaf set to $\mathcal{L}(T_{1})$. We proceed by adapting a proof
that pairwise compatible (unrooted) splits can always be combined
to form a tree. The adaptation is necessary because the splits of
$T_{1}$ and $T_{2}$ have different leaf sets. We can handle rootedness
in the unrooted context by simply treating the root as a special leaf.
Now, let us consider a split $\sigma$ of $T_{1}$ that is not implied
by any branch of $T_{2}$. Then $\sigma$ induces a flow from node
to node on $T_{2}|_{\mathcal{L}(T_{1})}$ (i.e. $T_{2}$ restricted
to $\mathcal{L}(T_{1})$) by assigning a direction to each branch
of $T_{2}|_{\mathcal{L}(T_{1})}$. That is, for each branch of $T_{2}|_{\mathcal{L}(T_{1})}$,
$\sigma$ is either on the left side or the right side of that branch.
An important fact (not shown here, but not very hard) is that each
node of $T_{2}|_{\mathcal{L}(T_{1})}$ has either all branches pointing
towards the node under the flow, or at most one branch pointing away.
Therefore, starting from any node, we may follow the flow from node
to node along the unique branch directed away from a node. Since this
is a flow on a tree, we may not have cycles. Therefore, there must
be a fixed point, which occurs when no branches point away from a
node. A second important fact (also not shown here) is that when we
find a node where all branches point towards to the node, then we
may impose the split $\sigma$ by adding a branch dividing some of
the branches at that node from other branches. This completes the
proof that if $\sigma$ is consistent with $T_{2}|_{\mathcal{L}(T_{1})}$
then we may add $\sigma$ to that tree.

However, in order to add $\sigma$ to $T_{2}$, we must adapt the
proof. Specifically, the node of $T_{2}|_{\mathcal{L}(T_{1})}$ where
we insert $\sigma$ is also a node of $T_{2}.$ However, whereas $\sigma$
partitions each branch of $T_{2}|_{\mathcal{L}(T_{1})}$ to one side
or the other side of the newly inserted branch, there may be branches
at the node of $T_{2}$ that contain no taxa from $T_{1}$. These
branches are thus not partitioned to one side or the other side of
the new branch. We solve this ambiguity by attaching them to the rootward
side of the new branch. However, in reality the branch imposed by
$\sigma$ be part of a connected collection of branches over which
these clades of $T_{2}$ may wander. Note that not all taxa in $\mathcal{L}(T_{2})-\mathcal{L}(T_{1})$
may wander across the branch imposed by sigma.

Sigh. Clearly we need some pictures here.


\section{Questions}
\begin{itemize}
\item Is there a single, unique solution to this problem?
\item When running BUILD, is it possible to construct a \emph{reason} for
the lack of inclusion of each split? Specifically, can we say which
split (or set of splits) conflicts with that split?

\begin{itemize}
\item We can construct the graph such that each internal node of a tree
is a vertex that connects to the vertex of its children. Instead of
cutting branches, we can remove vertices (and their connected edges).
(Unlike Stephen's graph, internal nodes from different trees should
\emph{not }be merged, I think. At most, we could record that an internal
node vertex has multiplicity 2, but each internal node is also going
to be associated with a particular leaf set, and it unlikely that
the leafs sets would be identical.) 
\item If we cut the vertex corresponding to a conflicting split, this should
be allow the BUILD algorithm to proceed, \emph{a la }Semple and Steel's
mincut supertree method.\end{itemize}
\end{itemize}

\end{document}
