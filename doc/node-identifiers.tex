\documentclass[11pt]{article}
\usepackage{amssymb}
\usepackage{amsmath}
\usepackage{pdfpages}
\usepackage{graphicx}
\usepackage[authoryear,round]{natbib}
\usepackage{algorithm,algorithmic}
\bibliographystyle{plainnat}
\usepackage{color}
\usepackage{graphicx}
\newtheorem{theorem}{Theorem}
\newtheorem{lemma}[theorem]{Lemma}
\newtheorem{corollary}[theorem]{Corollary}
\newtheorem{conjecture}[theorem]{Conjecture}
\newtheorem{definition}{Definition}
\usepackage{xspace}
\usepackage{paralist}
\usepackage{setspace}
\usepackage{wrapfig}
\textwidth 6.5in
\textheight 9in
\hoffset -1in
\voffset -1.4in
\singlespacing
\parindent 0em
\parskip .4em
% Abbreviations
\newcommand{\otol}{Open Tree of Life\xspace}
\newcommand{\ps}{phylogenetic statement\xspace}
\newcommand{\pss}{phylogenetic statements\xspace}
\newcommand{\PSs}{Phylogenetic Statements\xspace}
\newcommand{\PS}{Phylogenetic Statement\xspace}
\newcommand{\SWIPSD}{Sum of Weighted Input \PSs Displayed\xspace}
\newcommand{\MSWIPSD}{Maximum \SWIPSD \xspace}
\newcommand{\newick}[1]{\texttt{#1}\xspace}
\newcommand{\otc}[0]{\texttt{otcetera}\xspace}
\newcommand{\otcprune}[0]{\texttt{otc-prune-taxonomy}\xspace}
\newcommand{\otcdecompose}[0]{\texttt{otc-uncontested-decompose}\xspace}
\newcommand{\nexson}[0]{\texttt{otNexSON}\xspace}
\newcommand{\gcmdr}[0]{\texttt{gcmdr}\xspace}
\newcommand{\simplification}[0]{\\\noindent\textsc{Simplification Action(s)}:\xspace}
\newcommand{\undoActions}[0]{\\\noindent\textsc{``Undo'' Simplification Action(s)}:\xspace}
\newcommand{\stepExplanation}[0]{\\\noindent\textsc{Explanation}:\xspace}
\newcommand{\stepInput}[0]{\\\noindent\textsc{Input}:\xspace}
\newcommand{\stepOutput}[0]{\\\noindent\textsc{Output}:\xspace}
\newcommand{\currImpl}[0]{\\\noindent\textsc{Current Impl.}:\xspace}
\newcommand{\implTODO}[0]{\\\noindent\textsc{TODO for Impl.}:\xspace}
\newcommand{\currURL}[0]{\\\noindent\textsc{URL for output}:\xspace}
\newcommand{\comment}[1]{{\color{red} \textsc{#1}}\xspace}
\newcommand{\TODO}[1]{\comment{TODO: #1}}
\newcommand{\NeedsAlgorithmicWork}{{\comment{This needs algorithmic work.}}}
\newcommand{\ProofWriteupNeeded}{{\comment{Need to write up this proof}}}

\newcommand{\incLSSS}{Include-LeafSet support statement\xspace}
\newcommand{\incLSSSs}{Include-LeafSet support statements\xspace}

% Notation
\newcommand{\pssInOptimalTree}{\ensuremath{\hat{\mathcal{G}}}\xspace}
\newcommand{\pssFrom}[1]{\ensuremath{\mathcal{G}(#1)}\xspace}
\newcommand{\tripleSetInOptimal}{\ensuremath{\hat{\mathcal{R}}}\xspace}
\newcommand{\leafLabels}[1]{\ensuremath{\mathcal{L}(#1)}}
\newcommand{\parent}[1]{\ensuremath{\mbox{parent}(#1)}}
\newcommand{\children}[1]{\ensuremath{\mbox{children}(#1)}}
\newcommand{\nodes}[1]{\ensuremath{\mbox{nodes}(#1)}}
\newcommand{\treeRoot}[1]{\ensuremath{\mbox{root}(#1)}}
\newcommand{\taxonomy}[0]{\ensuremath{\mathbb{T}}\xspace}
\newcommand{\prunedTaxonomy}[0]{\ensuremath{\mathbb{T}_P}\xspace}
\newcommand{\phyloInputs}[0]{\ensuremath{\mathcal{T}}}
\newcommand{\expandedPhylo}[0]{\ensuremath{\mathcal{T}_{E}}\xspace}
\newcommand{\prunedSummary}[0]{\ensuremath{\mathcal{S}_{P}}\xspace}
\newcommand{\summaryTree}[0]{\ensuremath{\mathcal{S}}\xspace}
% verbatim, verbatim notation for a \ps
\newcommand{\vvps}[2]{\ensuremath{{#1}\downarrow{#2}}}
% leaf set, verbatim notation for a \ps
\newcommand{\lvps}[2]{\ensuremath{\leafLabels{#1}\downarrow{#2}}}
\newcommand{\leafComp}[2]{\ensuremath{\widetilde{\mathcal{L}_{#2}}\left({#1}\right)}}
\newcommand{\displaysPred}[2]{\ensuremath{\mathbb{I}_d(#1, #2)}}

\newcommand{\leafDes}[1]{\ensuremath{\mathcal{L}_d(#1)}}
\newcommand{\excLeafDes}[1]{\ensuremath{\mathcal{L}_e(#1)}}
%\newcommand{\children}[1]{\ensuremath{\mathcal{C}(#1)}}

\newcommand{\mrca}[2]{\ensuremath{\mbox{mrca}(#1, #2)}}
\newcommand{\treeOf}[1]{\ensuremath{\mbox{tree}(#1)}}
\newcommand{\leafSet}[1]{\leafLabels{#1}}}

\newcommand{\supportingPSSetFull}[2]{\ensuremath{\mbox{\texttt{SuppStatementSet}}(#1, #2)}}
\newcommand{\supportingPSSet}[1]{\ensuremath{\mbox{\texttt{SSS}}(#1)}}
\DeclareMathOperator*{\argmax}{\arg\!\max}

\usepackage{hyperref}
\hypersetup{backref,  linkcolor=blue, citecolor=black, colorlinks=true, hyperindex=true}
\begin{document}
The source for this in the doc subdirectory of the otcetera
    repo \url{https://github.com/mtholder/otcetera/tree/master/doc}.
\begin{center}
    {\bf Proposals for node identification systems in the Open Tree of Life project} \\
{Mark T.~Holder$^{1,2,\ast}$. feel free to contribute and add your name}
\end{center}
This doc really has nothing to do with otcetera, MTH just wanted a spot to
put a \LaTeX doc on this subject, and this repo had a convenient template.

\tableofcontents
\section{Background}
As of November 2015 nodes in the synthetic tree are identified in API calls
    using unstable neo4j node IDs.
If these node IDs resolve across versions of the synthetic tree, they will be
    resolving to an arbitrary spot in the synthetic tree.
Node IDs show up in URLs (some of which users may be in users' bookmarks, hence
    are out of our control) and in our commenting system (feedback repo on GitHub).

We also have IDs for taxa in our reference taxonomy, OTT.
That taxonomy is created by rule-based merging of multiple source taxonomies and
    a set of ``patch files.''
The system for deciding when a taxon has changed enough to merit a different OTT Id
    is automated, but not described in user-facing documentation.
In other words, we have not made any promises to people about what these IDs mean.

We would like to accelerate the rate of producing new synthetic trees.
That would exacerbate the problems with node-base URLs.
It would also be nice to be able to answer questions like ``how long has this grouping
been in the synthetic tree? (i.e. what version of the tree did this group first appear''

Any proposal for a system of generating node IDs would have to describe
\begin{compactenum}
  \item how the
    set of these Ids gets updated when a new version of the synthetic tree or taxonomy are produced,
  \item how changes in the id would affect the URLs from a previous version of the tree, 
  \item how the changes of Ids would affect the phylesystem corpus of studies that 
  are mapped to OTT names (and store the taxon name and OTT Id in the versioned NexSON
  files), and
  \item what we can promise users of Open Tree about these ids.
\end{compactenum}
\bibliography{otcetera}
\end{document}
