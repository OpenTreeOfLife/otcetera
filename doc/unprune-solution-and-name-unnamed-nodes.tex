\documentclass[english]{article}
\usepackage[T1]{fontenc}
\usepackage[latin9]{inputenc}
\usepackage{geometry}
\geometry{verbose,tmargin=2cm,bmargin=2cm,lmargin=2cm,rmargin=2cm}
\usepackage{babel}
\usepackage{xspace}

\newcommand{\parent}[1]{p(#1)}
\newcommand{\tripartition}[1]{\left[{#1}_I \mid {#1}_E \updownarrow {#1}_N\right]}
\newcommand{\incsed}[0]{{\em incertae sedis}\xspace}
\usepackage{hyperref}
\hypersetup{backref,  linkcolor=blue, citecolor=black, colorlinks=true, hyperindex=true}
\begin{document}

\title{\texttt{otc-unprune-solution-and-name-unnamed-nodes}}

\maketitle

\tableofcontents

\section{Command line interface}
This tool takes 2 mandatory arguments:
\begin{enumerate}
    \item a filepath to a Newick representation of the grafted supertree
    \item a filepath to a Newick representation of the ``cleaned'' taxonomy (the taxonomy
    without flagged taxa).
\end{enumerate}
Options include:
    \item[]{\tt-i$<$filepath$>$} to specify the input filepath for file with OTT IDs of
    the \incsed taxa listed (one per line). The IDs of such taxa 
    in the exemplified taxonomy is sufficient.
    \item[]{\tt-j$<$filepath$>$} to specify an output filepath for the broken taxon JSON.
    \item[]{\tt-s$<$filepath$>$} to specify an output filepath for a statistics JSON file.
\end{itemize}
The tool writes a Newick representation of the full tree with internal node labels to 
standard output.

The propinquity invocation is essentially:
\begin{verbatim}
otc-unprune-solution-and-name-unnamed-nodes \
  grafted_solution/grafted_solution.tre \
  cleaned_ott/cleaned_ott.tre \
  -iexemplified_phylo/incertae_sedis.txt \
  -jlabelled_supertree/broken_taxa.json \
  -slabelled_supertree/input_output_stats.json \
  > labelled_supertree/labelled_supertree.tre
\end{verbatim}


\section{Notation and Background}
\begin{itemize}
    \item $G$ will denote the input supertree
    \item $U$  denotes the final output tree.
    \item $\mathcal{C}$ denotes the set of higher taxa in the taxonomic tree
        that are compatible with $G$.
    \item for any nonmonotypic, sampled taxon $Y$:
    \begin{itemize}
        \item $M_G(Y)$ denotes the node in input supertree that is the MRCA of
            the any taxon that is sampled in the supertree and is in the
            include group of the taxon $Y$.
        \item $M_U(Y)$ denotes the MRCA of the include group of $Y$ in the final tree.
        \item The algorithm act through $J$ steps, indexed by $j$.
            $M_j(Y)$ refers to the MRCA fo $Y$ on the supertree at step $j$.
    \end{itemize}
    \item for any monotypic taxon $Y$, and using $x$ is a stand-in for $G$, $U$, or $j$ in the previous notation, we will say:
    \begin{itemize}
        \item if $Y\notin\mathcal{C}$, then $M_x(Y)$ refers
            to $M_x(Z)$ where $Z$ is the first non-monotypic descendant of $Y$
        \item if $Y \in \mathcal{C}$, then $M_x(Y)$ refers
            to the parent of $M_x(Z)$ where $Z$ is the child of $Y$.
    \end{itemize}
    \item The parent node of node $Y$ will be denoted $\parent{Y}$.
\end{itemize}

\subsection{Labeling nodes -- taxa as tripartitions of the leaf set}
For naming node $X$, the tool uses the idea that each higher taxon, $Y$, in OTT can
    be described as a tripartition of the leaf set of OTT into:
\begin{enumerate}
    \item $Y_I$ an include set of taxa that must be descendants of the node $X$ (if they sampled in the tree)
    \item $Y_E$ an exclude set of taxa that cannot be descendants of the node $X$ (if they are sampled in the tree), and 
    \item $Y_N$ a non-excluded set which can fall within or outside of the group $X$; these taxa
    correspond to \incsed taxa that are children of one of the ancestors of $Y$.
\end{enumerate}
%A taxonomic tripartition will be denoted $\tripartition{Y}$.

The OTT ID $Y$ applies to node $X$ of tree $T$ if:
\begin{itemize}
    \item At least one member of $Y_I$ is sampled in tree $T$,
    \item node $X$ satisfies the constraints about these tip taxa, and
    \item node $X$ is the least inclusive internal node that is an ancestor all of the taxa from $Y_I$
\end{itemize}

\subsection{Goals of propinquity}
See \url{https://peerj.com/articles/3058/} for a more thorough discussion of the pipeline.
The overall goals are to produce a summary tree that displays as many input groupings as
    possible, only contains groupings that are supported by at least one
    input group (in the sense that dropping the grouping from the summary tree will cause
    the tree to no longer display the input group), and which prefer groups from 
    high ranked tree in a ranking scheme that puts the taxonomy as the lowest ranked
    input.
The unprune tool fits into this pipeline near the end.
When used by propinquity, the supertree input to the unprune tool has no
    groups that are unsupported.
But the input supertree only contains leaves that are represented in at least one 
    phylogenetic input tree.
These are referred to as the ``sampled'' taxonomic tips.
The job of the tool is to add the rest of the tips from the taxonomy to the tree.

The goal is to do this while displaying as many input groups as possible.
Since the phylogenies always overrule the taxonomy, the groupings of the input
    tree cannot be altered by considering the taxonomy.
Hence, the input tree is a backbone constraint for the output tree.
The tool should display one of the possible full supertrees which displays as
    many higher taxa as possible in order to fulfull the pipeline's goals of
    maximizing the number of groupings displayed.




\section{Inputs}
\subsection{input supertree}
The grafted supertree has OTT IDs on every tip and some internal nodes.
Minimally the root of each subproblem should have an OTT ID.
If a taxon was contested, but still recovered in the subproblem solution it might
have an OTT ID.  \footnote{This would be fine if it is labelled. I am not sure
if the subproblem solver emits labels in such cases. BEN: It does.}.


The algorithm will be more efficient if as many internal nodes that can have IDs are 
    assigned IDs on input.
The software may crash or emit an incorrect answer if the IDs of the internal node
    of the input tree are invalid\footnote{Haven't checked this statement - I know it assumes the labels are right. Not sure what happens if they are wrong.}. 
A valid label of an internal node with a taxon ID means that the node is the MRCA
    of the constituent subtaxa and that the node is not the ancestor of any
    excluded taxon.

As mentioned above, the unprune tool assumes that each internal edge in this tree
    is supported by at least one input.
Thus, the tree is treated as a backbone constraint for the solution.

\subsection{taxonomy tree}
The taxonomy must have a superset of the OTT IDs mentioned in the grafted supertree.
The tree may contain polytomies and nodes with out-degree of 1.
Each OTT ID mentioned in the tree must be unique.
Every internal node of the tree must be labelled with an OTT ID.

\subsection{list of \incsed taxa}
Each ID should correspond to an OTT ID that labels an internal node in the taxonomy tree.
Any taxon that is in the subtree rooted at a \incsed taxon is
    a member of the ``non-exclude'' set of every taxon that is in one
    of the subtrees that are rooted by the siblings of the \incsed taxon.

\section{Algorithm}
\subsection{Summary}
Essentially, we consider each taxon in the taxonomy to see if it can be added to the grafted tree.
If the taxon is compatible with the tree it will either:
\begin{itemize}
    \item result in an existing node in the supertree being labeled with the OTT ID, or
    \item result in a new node being added to the supertree.
\end{itemize}
If the taxon is not compatible with the supertree, then an entry in the ``LostTaxonMap'' data
    structure will be made; this data structure will store pointers to the
    MRCA of members of the taxon as well as attachment points for subtaxa onto the tree, and
    IDs of the other taxa that have ``intruded'' to break up the taxon.
This LostTaxonMap is serialized as the broken taxon JSON, if the user requested that output.


The search for the tree that displays as many of the taxonomy nodes as possible is not difficult.
A subset of higher taxa, $\mathcal{C}$ are compatible with the input supertree as a backbone.
Each higher taxon that is not in this group is incompatible with the input.
It turns out that, by careful placement of the unsampled taxon, we can add every 
    member of $\mathcal{C}$ to the backbone.
Thus a greedy search displays all of the groupings that could conceivably be placed on the tree.

\subsection{Conceptualization of the steps}
Nodes are actually moved from the taxonomic tree to the supertree during execution.
If the taxonomy tree has $J$ internal nodes, then we can think of the process as being
a transformation of the trees:
\[M_G(Y) = M_0(Y) \rightarrow M_1(Y) \rightarrow M_2(Y) \rightarrow \ldots M_{J-1}(Y) \rightarrow M_J(Y) = M_U(Y).\]

We process the higher taxa in postorder fashion (moving from near the tips toward root) dealing with one higher taxon at each step.
At step $j$ we deal with taxon $Y_j$.

A higher taxon can be classified based on how many of its children are
sampled in the input supertree (where ``sampled'' means either occuring as a tip label in the tree or having a descendant that was a tip label in the tree).

\subsubsection{unsampled taxa}
If taxon $Y_j$ has 0 of children that were sampled, then $Y_j\in \mathcal{C}$.
In this case, $Y_j$ can simply be added to the supertree as a child of the supertree node that maps to $\parent{Y_j}$.
Let $\parent{Y_j} = Y_k$.
Note that the postorder traversal guarantees that $k > j$.
Because $Y_j$ and all of its children will be grafted on the supertree at step $k$ as
an entire subtree, $Y_j$ will not be represented on the supertree before step $k$.
But at step $k$, the supertree will display $Y_j$.
None of the subsequently handled taxa (for any index $l> k$) will 
graft taxa inside of $M_l(Y_j)$, so every subsequent tree including the
final tree will still display $Y_j$.

\subsubsection{multiply-sampled taxa}
For brevity, let $W_j = M_{j-1}(Y_j)$.
This node is the MRCA of the sampled subtaxa of $Y_j$ in the previous step.
If multiple children of $Y_j$ were sampled in the input tree, then this guarantees that MRCA is an internal, forking node.

\noindent{\bf The incompatible case}\\
If $W_j$ is an ancestor of any taxon that is in the exclude group of $Y_j$, then $Y_j\notin\mathcal{C}$.
In this case, we attach all of the unsampled children of $Y_j$ as children of $W_j$.
This placement of additional taxa will not cause any subsequently handled taxa to move from
    the ``compatible with the supertree'' category to the ``incomatible with the supertree.
We can see this because adding the unsampled taxa increases the size of the include group
    of $M_{k}(Y_j)$ and all of its ancestors for this step and all later steps (in other words
    for $k \geq j$).
Every taxon for $k>j$ will either be an ancestor of $Y_j$ or in a separate subtree; {\em i.e.~} not a descendant of $Y_j$.
If the subsequent taxon $Y_k$ is an ancestor of $Y_j$, then adding more children of $Y_j$ will not cause $Y_k$ to be incompatible.
The fact that $W_j$ was the MRCA of some children of $Y_j$ guarantees that that $M_{k-1}(Y_k)$
    would have to be $M_{k-1}(Y_j)$ or one of its ancestors.
Adding more children does not change which node will be the ancestory of $Y_k$ when that taxon is handled.

If $Y_k$ is in a disjunct subtree of the taxonomy from $Y_l$, then its compatibility 
status will also be unaffected by adding the previously unsampled descendants of
$Y_j$ to $W_j$.

The members of $Y_j$ could be in the non-exclude set of $Y_k$, but this can only happen if
    $Y_j$ is itself \incsed.
In this case all of the descendants of $Y_j$ are irrelevant with respect to the compatibility
    of $Y_k$ with the supertree; so adding more descendants does not affect $Y_k$

Alternatively, the taxa descended from $Y_j$ could be in the exclude set of $Y_k$.
In this case, $Y_k$ would be incompatible with the supertree if $M_{k-1}(Y_k)$ is
    $M_{k-1}(Y_j)$ or one of is one of its ancestors.
Once again, the fact that adding unsampled taxa does not change the identity of $M_{k-1}(Y_j)$
    means that the addition cannot invalidate $Y_k$ in this way.
If $M_{k-1}(Y_k)$ is not one of the nodes on the path from $M_{k-1}(Y_j)$ to the root, then
    every member of $M_{k-1}(Y_j)$ will be represented as part of the exclude group
    of the node $M_{k-1}(Y_k)$.
Clearly adding excluded taxa to the exclude group of the MRCA of $Y_k$, won't invalidate the 
compatability of $Y_k$. 

\noindent{\bf The compatible case}\\
If $W_j$ is not an ancestor of any taxon that is in the exclude group of $Y_j$, then $Y_j\in\mathcal{C}$.
Adding all of the unsampled children of $Y_l$ as children of $W_j$ would succeed in displaying
    the taxon $Y_j$ on the tree.

Importantly though, this step could invalidate an existing taxon, $Y_m$ where $m < j$ and $W_j = M_{j-1}(Y_m)$.
This can happen if $Y_j$ has one sampled non-\incsed child, then all of the other sampled children must correspond to \incsed taxa.
In this case, $W_j$ will already be labelled with an OTT ID (the ID of its non-\incsed child).
Adding more unsampled children to $W_j$ could either invalidate the current label of
    $W_j$ (if those children were not \incsed taxa) or cause the supertree
    to display only $Y_j$ instead of both $Y_j$ and the taxon that currently is mapped to $W_j$.
The unprune tool checks for this case by checking for an OTT ID assignment to $W_j$.
If the node already has an OTT ID, and the identified taxon is a descendant taxon of $Y_j$, 
then we need to add a new node as a direct ancestor of $W_j$, and attach the unsampled 
children of $Y_j$ to that node.
That node will be $M_j(Y_j)$.

It is also possible because of intrusion of non-excluded taxa, that $M_j(Y_j)$ is already 
validly assigned a labelling OTT ID, but adding a new parent node would not add a supported
node.
This happens when the OTT taxon that labels $W_j$ is not a descendant of $Y_j$.
In this case the unprune tool notes the synonymy.



\subsubsection{singly-sampled taxa}
If exactly 1 of child of $Y_j$ was sampled, call this child $Y_m$ (where $m < j$ 
    is guaranteed by the postorder traversal).
Then the handling of $Y_j$ depends on 
    whether $Y_m\in\mathcal{C}$ or $Y_m\notin\mathcal{C}$.

\noindent{\bf The incompatible case}\\
If $Y_m\notin\mathcal{C}$, then $M_{j-1}(Y_m)$ is also the ancestor
    of some taxa that in the exclude group of $Y_m$.
These taxa are also in the exclude group of $Y_j$.
We can see via the following argument.
Let the excluded intruder be $X$ ($X$ is in the exclude group of $Y_m$)
Note that the non-exclude group of $Y_m$ is a superset of the non-exclude group
    of $Y_j$.
Thus, it is impossible for $X$ to be in the non-exclude group of $Y_j$.
The exclude group of $Y_m$ is the union of the exclude group of $Y_j$ and the include
    group of every non-\incsed sibling of $Y_m$ in the taxonomy.
However, because $Y_m$ is the only child of $Y_j$ that was sampled, it is 
    not possible that $X$ is a descendant of $Y_j$.
Because $X$ is not a descendant of $Y_j$, and is not in $Y_j$'s non-exclude group,
    $X$ must be in the exclude group of $Y_j$.

This establishes that $Y_j\notin\mathcal{C}$.
As in the incompatible case for the multiply-sampled taxa, we attach
    the unsampled children of $Y_j$ as children of $M_{j-1}(Y_m)$ and note that 
    the taxon is broken in the supertree.
For the same reason as discussed above, this attachment will not invalidate
    any taxon that is in $\mathcal{C}$.

\noindent{\bf The compatible case}\\
This case acts just like the compatible case for the multiply-sampled taxon in which
    the relevant node in the supertree has the label of the child of $Y_j$.
In other words, we find $M_{j-1}(Z)$ where $Z$ is the child of $Y_j$.
This node will be labeled with the ID of $Z$, because $Y_j\in\mathcal{C}$ if and
    only if $Z\in\mathcal{C}$.
Then we bisect the edge between $M_{j-1}(Z)$ and its parent by adding a new node. 
This parent now meets the definition of $M_{j}(Y_j)$, and we assign it the ID from
    $Y_j$.

\subsubsection{Summary of the conceptual description}
The previous sections are intended to show that a fairly simple recipe of:
\begin{enumerate} 
    \item traversing the taxonomy in an order that guarantees that a taxon 
        is processed after its children;
    \itme skipping a taxon if it is unsampled;
    \item finding the MRCA of the include group of the taxon in the supertree;
    \item deciding whether a focal node $X$ for the taxon is the MRCA or a newly
        created node that bisects the edge from the MRCA to its parent; and
    \item attaching all unsampled children of the taxon to $X$ and possibly
        labeling $X$ with the taxon's ID
\end{enumerate}
will succeed in adding all of the taxa that are in $\mathcal{C}$ to the supertree.

\section{Tricks used in the actual implementation}
The problem with a naive implementation of the conceptual description of the 
    algorithm, is that detecting compatibility easily requires use of
    the sets of descendants for each node (the \texttt{des\_ids} field of
    a \texttt{TreeNode} in otcetera).
This would be use a lot of memory when done on the full supertree and taxonomy.

\subsection{Preprocess to glean \incsed info}
The \texttt{register\_incertae\_sedis\_info} function uses the \incsed ID list
    and the \texttt{des\_ids} field of the taxonomy to:
\begin{itemize}
  \item Note which IDs are in the nonexcluded IDs field for each higher taxon.
    Note that if none of a taxon's children are \incsed, then each of its
    children will have the same set of nonexcluded IDs as the taxon itself.
    Thus, we use a pointer for \texttt{nonexcluded\_ids} so that all of the 
    nodes that have the same set of IDs can simply alias the same \texttt{set}.
  \item Fill in set of sampled tips in the supertree that are descendants of
    an \incsed taxon. This set is stored in \texttt{supertree\_tips\_des\_from\_inc\_sed}.
  \item Fill in a map of from a pointer to a taxonomy node that is either \incsed or a descendant
    of an \incsed taxon to the set of sampled supertree tips. This is
    the \texttt{inc\_sed\_taxon\_to\_sampled\_tips} map.
  \item Fill in a map from each nonmonophyletic \incsed taxon's ID to the
    \texttt{des\_ids} set of descendants.
    This is not done in an efficient manner currently.
    We will re-use the \texttt{des\_ids} field of the taxonomy nodes in a subsequent
    step, so we have to cache away this set so that we can make an accurate LostTaxonMap
    for these taxa later. The map is called \texttt{nonmonophyletic\_inc\_sed\_to\_des\_ids}
\end{itemize}
At the completion of this step, the \texttt{des\_ids} fields of the taxonomy are
    cleared for reuse in the next step.

\subsection{Unpruning via slices of the supertree}
Imagine cutting the supertree at every node that has an OTT ID, and
   assigning a copy of the cut nodes to both the tipward tree (as root)
   and to rootward tree (as a tip).
The implementation does not detach the nodes to make these cuts, but
    it does process the supertree in what are referred to as ``slices''
    that correspond to the trees that would be produced by this cutting.

The supertree is processed in postorder, and whenever an internal node with an OTT ID is
    encountered, that node is treated as the root of a slice by calling \texttt{unprune\_slice}.
Moving tipward from the root of the slice, the first node with an OTT ID encountered down
    each branch serves as the leaf set for the slice.
So the OTT IDs of the supertree serve as signals of a slice boundary.


Crucial to the performance is that the fact that the \texttt{des\_ids} fields of the supertree
    are only filled in within a slice. 
After the slice is processed, these fields are cleared and the \texttt{des\_ids} field
    for the root is set to simply contain the OTT ID of that taxon.
Thus, that node acts just like a leaf (only has one element in \texttt{des\_ids} field).


Within \texttt{unprune\_slice}, the tips IDs for the slice are identified.
The corresponding taxa are found in the taxonomy tree.
The \texttt{des\_ids} field of induced taxonomic tree are filled in using the sampled
    ``tips'' IDs only.
Knowing the taxonomic root of the slice (because we know the ID of the root of the
    slice makes this easy to do).
Note that ``tips'' is in quotes because these are just acting like tips of the slice
    (based on their possession of an OTT ID); the nodes are just effectively tips
    for this function.

Next we process the taxonomic nodes of this induced tree in postorder.
\texttt{detect\_leaves\_for\_slice} returns the set of node pointer for
    the slice for both the taxonomy and the supertree,
    and reversing the vector of nodes returned by 
    \texttt{preorder\_below\_boundaries} gives the traversal of taxonomy 
    nodes without creating a separate tree data structure for the induced tree.

The possibility of having \incsed taxa that are broken and occurring in more than
    one slice complicates this whole procedure.
So a fair amount of bookkeeping is done in \texttt{unprune\_slice} to 
    know which taxonomy ``tips'' are from \incsed taxa and whether or 
    not the rest of the taxon is in a lower part of the tree.

The \texttt{incorporate\_higher\_taxon} performs the decision making about whether
    the taxon is broken, whether its corresponding node in the supertree should
    get that taxon's ID, whether a new node should be created to bisect an edge in
    the existing supertree.
It also moves nodes that a the roots of unsampled subtrees from the taxonomy
    tree to the supertree.

A taxonomy node can be recognized as unsampled by having an empty \texttt{des\_ids}
    field and having \texttt{repr\_in\_supertree\_by} set to be a null pointer.
This is the initial state for all nodes.
But when a floating (\incsed or descendant) taxon is in a higher slice in the tree
    this field will point to that slice's root.
This is necessary to prevent grafting on the \incsed taxon onto the tree so that it
    occurs a second time.

The tipward elements of multi-slice \incsed taxa present another challenge.
The are excluded in the process of finding the induced taxonomic tree.
They must be excluded in that step because they are not descendants of the root taxon.
Instead we mark the \texttt{des\_ids} fields of these rootward until
    we run the ancestral taxa stop being \incsed-flagged.
Separate data structures are needed to keep track of which \incsed are sampled
    in the current slice.
The depth of the taxonomic nodes is used to assure that any higher \incsed taxa
    that should be handled in the slice are handled in post order.

\end{document}
