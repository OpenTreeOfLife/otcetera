\newcommand{\defitem}[2]{\item[{\bf #1}]\label{itm:#2} }
\newcommand{\notitem}[1]{\item[]#1}
\section{Glossary}
\begin{compactenum}
\defitem{as complete}{defAsComplete} a phrase contrasting 2 trees.
    Tree $A$ is as complete as tree $B$ $\iff$
    leaf label set of $B$ is a subset of the leaf label set of $A$.\\
    In notation: $\leafLabels{B} \subseteq \leafLabels{A}$.
\defitem{cluster}{defCluster} the set of leaf nodes that are descendants
    of a node, $x$, is the cluster of $x$
\defitem{complete}{defComplete} as an adjective for a tree. Tree $A$ is complete $\iff$ the leaf label set
    of tree $A$ is identical to the the set of terminal taxa.
    In notation: $\leafLabels{A} = \leafLabels{\taxonomy}$
\defitem{display}{defDisplay} Tree $A$ displays a node, $x$, from tree $B$ $\iff$
    that $A$ is as complete as tree $B$ and the induced tree of $A$ limited to the 
    leaf label set of $B$ contains a cluster with labels identical to the label set of the cluster of $x$.\\
    An equivalent characterization in terms of the full leaf set of $A$ would be:
    tree $A$ is as complete as $B$ and there is a node in $A$ which has a leaf label set 
    that is a superset of \leafLabels{x} and which does not contain any member of $\leafLabels{B} \setminus \leafLabels{x}$.\\
    We say that a tree $A$ displays a \ps $x$, if you were to translate
        $x$ tree with single internal node $y$, and tree $A$ displays $y$.
\defitem{internal node}{defInternalNode} a node that is not the root or a leaf.
\defitem{label}{defLabel} When speaking of the label of a node in this document, we are referring to unique taxonomic
    identifier for that node.
    This document does not discuss any issues associated with mapping strings to taxonomic names.
\defitem{leaf labels}{defLeafLabels} For our problems, the input trees have been aligned to a common
    taxonomy. 
    So referring to the leaf label set of a tree can be interpreted as
    ``the set of taxonomic identifiers that are mapped to the leaves of the tree''.
    The leaf label set of a node is the set of labels associated with the cluster of the node.\\
    Note that a leaf label is not necessarily a terminal taxon.
    In notation, $\leafLabels{x}$ is the leaf label set of $x$.
\defitem{more complete}{defMoreComplete} tree $A$ is {\em more complete} than tree $B$ $\iff$ the leaf
    label set of $B$ is a proper subset of the leaf label set of $A$. \\
    In notation, $\leafLabels{B} \subset \leafLabels{A}$.
\defitem{phylogenetic statement}{defPS} (we are considering ``rooted split'' or ``rooted bipartition'' for this phrase)
    This is a pair of sets of taxonomic id sets: the include set and the exclude set (called ``ingroup'' and ``outgroup''
    on the google doc).\\
    The interpretation of a \ps is: if the statement is true, then 
    any pair of elements in the include are more closely related to each
        other than they are to any element in the the exclude set.
    Thus if there are $N$ labels in the include set, and $M$ in the exclude set, the \ps implies
    ${N \choose 2}M$ distinct rooted triples.\\
    The include set and the exclude set must be disjunct. Their union is referred to as the leaf label set
        for the \ps.
    When we use the phrase ``\ps'' without qualifier, we mean a nontrivial statement.
    For a \ps to be nontrivial there must be at least 2 elements in the include group and at least 
        1 in the exclude group.\\
    We use ``\vvps{\mbox{include}}{\mbox{exclude}}'' to denote a \ps. 
    This is very similar to split syntax in unrooted phylogenetics, but with the arrow 
        in place of $\mid$ to emphasize that the statement is oriented.\\
    Each internal node in a tree maps to a nontrivial \ps that can created by setting
        the include set equal to the leaf labels of node and setting the exclude set equal
        to the leaf label set of the tree minus the leaf labels of the node.
        In notation, node $x$ in tree $T$, maps to: $\lvps{x}{[\leafLabels{T} \setminus \leafLabels{x}]}$.\\
    A tree can be converted to a set of \pss: one for each internal node in the tree.
\defitem{terminal taxon}{defTerminalTaxon} a taxonomic identifier for at taxonomic node that
    has no children {\em in the taxonomic tree}.
\defitem{trivial}{defTrivial} a trivial \ps is one which must be found and any tree that has
    the leaf labels set of the \ps.
    This includes

\end{compactenum}
\section{Notation}
\begin{compactenum}
    \notitem{\leafLabels{x}} is the leaf label set of $x$.
    \notitem{\taxonomy} the taxonomic tree.
    \notitem{\leafComp{x}{T}} is the exclude group for node $x$ on tree $T$. It is the 
        set complement of \leafLabels{x} with respect to the 
    \notitem{\displaysPred{T}{x}} is an identity function that evaluates to 1 if tree $T$
        displays $x$, and to 0 otherwise.
\end{compactenum}
