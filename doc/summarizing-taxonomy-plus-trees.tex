\documentclass[11pt]{article}
\usepackage{amssymb}
\usepackage[authoryear,round]{natbib}
\bibliographystyle{plainnat}
\usepackage{bm}
\usepackage{color}
\usepackage{graphicx}
\newtheorem{theorem}{Theorem}
\newtheorem{lemma}[theorem]{Lemma}
\newtheorem{corollary}[theorem]{Corollary}
\newtheorem{definition}{Definition}
\usepackage{xspace}
\usepackage{paralist}
\usepackage{setspace}
\textwidth 6.5in
\hoffset -1in
\singlespacing

% Abbreviations
\newcommand{\otol}{Open Tree of Life\xspace}
\newcommand{\ps}{phylogenetic statement\xspace}
\newcommand{\pss}{phylogenetic statements\xspace}
\newcommand{\PSs}{Phylogenetic Statements\xspace}
\newcommand{\PS}{Phylogenetic Statement\xspace}

% Notation
\newcommand{\leafLabels}[1]{\ensuremath{\mathcal{L}(#1)}}
\newcommand{\taxonomy}[0]{\ensuremath{\mathbb{T}}}
% verbatim, verbatim notation for a \ps
\newcommand{\vvps}[2]{\ensuremath{{#1}\downarrow{#2}}}
% leaf set, verbatim notation for a \ps
\newcommand{\lvps}[2]{\ensuremath{\leafLabels{#1}\uparrow{#2}}}
\newcommand{\leafComp}[2]{\ensuremath{\widetilde{\mathcal{L}_{#2}}\left({#1}\right)}}
\newcommand{\displaysPred}[2]{\ensuremath{\mathbb{I}_d(#1, #2)}}

\usepackage{hyperref}
\hypersetup{backref,  linkcolor=blue, citecolor=black, colorlinks=true, hyperindex=true}
\begin{document}
The source for this in the doc subdirectory of the otcetera
    repo \url{https://github.com/OpenTreeOfLife/otcetera/tree/doc/doc}.
\begin{center}
    {\bf Summarizing a taxonomy and multiple estimates of phylogenetic trees} \\
{Mark T.~Holder$^{1,2,\ast}$. feel free to contribute and add your name}
\end{center}
\section{Background}
The \otol project is attempting to build a platform for summarizing what is known
    about phylogenetic relationships across all of Life.
Presenting an easy-to-interpret summary of trees that have been ``curated''
    is one component of that effort.
The project has decided that this summary should include a tree which
\begin{compactenum}
    \item can be served and browsed;
    \item contains annotations indicating which input trees support a particular grouping;
    \item has all of the tips of the taxonomy;
    \item displays as many of the groups in the input trees as is feasible;
    \item may utilize ranking of trees;
    \item (tentative - not sure if everyone is on board with this one) has no unsupported groups;\label{noUnsupportedReq}
    \item (tentative) does not ``prefer'' lack of resolution (defined more thoroughly below)
\end{compactenum}
This is results in a new form of a supertree problem described below as the ``taxonomy-based tree summary'' problem.
\subsection{Taxonomy-based supertree}
This is a novel name for a special form of the supertree problem.
A taxonomy-base supertree has at least one input (the taxonomic tree) is complete.
\subsection{Taxonomy-based summary tree}
Many supertree approaches seek to maximize accuracy according to some notion
    of distance between the true tree and estimated true.
The summary that we seek has requirements (e.g. \ref{noUnsupportedReq})
    which lead to less resolved trees.
Such trees may fail to display all of the well-supported (or even uncontested)
    rooted triples, but such trees also make it easier for users to see the
    connection between the summary tree and the input trees.
Thus, the phrase ``Taxonomy-based summary tree'' is used here to describe a
    taxonomy-based supertree which tries to maximize some notion of
    explaining the set of input trees (rather than a supertree designed to
    display the highest number of groups in the inputs, or some other criterion).

\subsection{Taxonomy-based summary of ranked input trees}
We have been pursuing a strategy that uses a ranking of trees.
In cases of conflict, the grouping that is compatible with the 
    higher ranked tree is shown in the tree (if it is not contradicted by 
    another grouping from trees of even higher rank).
 The taxonomic tree is considered to be the lowest ranked input.

Applying a ranking to the input trees is biologically questionable (because
    the importance of an input \ps should be based on the support for that
    grouping -- it is rarely the case that tree-wide rank would adequately
    describe the degree of statistical support for that group).
Using tree-based ranks also introduces subjectivity into the summary buildind process.

Nevertheless, tree-rank based summaries represent a reasonable starting point
    because they are easy to explain and the ranking permits some algorithmic
    simplifications.

\section{Details of some important concepts}
\subsection{The Sum of Weighted Input \PSs Displayed Score}
Let $\mathcal{P}$ be a multiset of \pss.
If each member, $i$, this set is assigned a weight, $w_i$, then the 
sum of weighted input \pss displayed score, $S_w$, for a tree $T$ is:
\begin{equation}
    S_w(T, \mathcal{P}) = \sum_{i\in \mathcal{P}} w_i \displaysPred{T}{i}
\end{equation}
where {\displaysPred{T}{x}} is an identity function that evaluates to 1 if tree $T$
        displays $x$, and to 0 otherwise.

Trying to find the set of trees that maximize $S_w$ is one natural
    goal for a summarization procedure.
We can call this the ``maximum sum of weighted input \pss displayed'' problem.

Tree-based ranking can be viewed as a means providing weights to \pss.
Each tree is converted to a set of \pss, and the tree's weight is 
    assigned to each \ps in the set.
The union of the tree's sets becomes the multiset, $\mathcal{P}$, referred to
    in the definition of the score above.

\url{https://github.com/OpenTreeOfLife/treemachine/wiki/MaxWeightOfInputTreeEdgesDisplayed}
    sketches out a proof of how an extreme form of tree-based ranks
    can lead to greedy tree addition strategy can be guaranteed to 
    find the set of trees that solve the maximimum sum of weighted input
    \pss displayed problem.



\newcommand{\defitem}[2]{\item[{\bf #1}]\label{itm:#2} }
\newcommand{\notitem}[1]{\item[]#1}
\section{Glossary}
\begin{compactenum}
\defitem{as complete}{defAsComplete} a phrase contrasting 2 trees.
    Tree $A$ is as complete as tree $B$ $\iff$
    leaf label set of $B$ is a subset of the leaf label set of $A$.\\
    In notation: $\leafLabels{B} \subseteq \leafLabels{A}$.
\defitem{cluster}{defCluster} the set of leaf nodes that are descendants
    of a node, $x$, is the cluster of $x$
\defitem{complete}{defComplete} as an adjective for a tree. Tree $A$ is complete $\iff$ the leaf label set
    of tree $A$ is identical to the the set of terminal taxa.
    In notation: $\leafLabels{A} = \leafLabels{\taxonomy}$
\defitem{display}{defDisplay} Tree $A$ displays a node, $x$, from tree $B$ $\iff$
    that $A$ is as complete as tree $B$ and the induced tree of $A$ limited to the 
    leaf label set of $B$ contains a cluster with labels identical to the label set of the cluster of $x$.\\
    An equivalent characterization in terms of the full leaf set of $A$ would be:
    tree $A$ is as complete as $B$ and there is a node in $A$ which has a leaf label set 
    that is a superset of \leafLabels{x} and which does not contain any member of $\leafLabels{B} \setminus \leafLabels{x}$.\\
    We say that a tree $A$ displays a \ps $x$, if you were to translate
        $x$ tree with single internal node $y$, and tree $A$ displays $y$.
\defitem{internal node}{defInternalNode} a node that is not the root or a leaf.
\defitem{label}{defLabel} When speaking of the label of a node in this document, we are referring to unique taxonomic
    identifier for that node.
    This document does not discuss any issues associated with mapping strings to taxonomic names.
\defitem{leaf labels}{defLeafLabels} For our problems, the input trees have been aligned to a common
    taxonomy. 
    So referring to the leaf label set of a tree can be interpreted as
    ``the set of taxonomic identifiers that are mapped to the leaves of the tree''.
    The leaf label set of a node is the set of labels associated with the cluster of the node.\\
    Note that a leaf label is not necessarily a terminal taxon.
    In notation, $\leafLabels{x}$ is the leaf label set of $x$.
\defitem{more complete}{defMoreComplete} tree $A$ is {\em more complete} than tree $B$ $\iff$ the leaf
    label set of $B$ is a proper subset of the leaf label set of $A$. \\
    In notation, $\leafLabels{B} \subset \leafLabels{A}$.
\defitem{phylogenetic statement}{defPS} (we are considering ``rooted split'' or ``rooted bipartition'' for this phrase)
    This is a pair of sets of taxonomic id sets: the include set and the exclude set (called ``ingroup'' and ``outgroup''
    on the google doc).\\
    The interpretation of a \ps is: if the statement is true, then 
    any pair of elements in the include are more closely related to each
        other than they are to any element in the the exclude set.
    Thus if there are $N$ labels in the include set, and $M$ in the exclude set, the \ps implies
    ${N \choose 2}M$ distinct rooted triples.\\
    The include set and the exclude set must be disjunct. Their union is referred to as the leaf label set
        for the \ps.
    When we use the phrase ``\ps'' without qualifier, we mean a nontrivial statement.
    For a \ps to be nontrivial there must be at least 2 elements in the include group and at least 
        1 in the exclude group.\\
    We use ``\vvps{\mbox{include}}{\mbox{exclude}}'' to denote a \ps. 
    This is very similar to split syntax in unrooted phylogenetics, but with the arrow 
        in place of $\mid$ to emphasize that the statement is oriented.\\
    Each internal node in a tree maps to a nontrivial \ps that can created by setting
        the include set equal to the leaf labels of node and setting the exclude set equal
        to the leaf label set of the tree minus the leaf labels of the node.
        In notation, node $x$ in tree $T$, maps to: $\lvps{x}{[\leafLabels{T} \setminus \leafLabels{x}]}$.\\
    A tree can be converted to a set of \pss: one for each internal node in the tree.
\defitem{terminal taxon}{defTerminalTaxon} a taxonomic identifier for at taxonomic node that
    has no children {\em in the taxonomic tree}.
\defitem{trivial}{defTrivial} a trivial \ps is one which must be found and any tree that has
    the leaf labels set of the \ps.
    This includes

\end{compactenum}
\section{Notation}
\begin{compactenum}
    \notitem{\leafLabels{x}} is the leaf label set of $x$.
    \notitem{\taxonomy} the taxonomic tree.
    \notitem{\leafComp{x}{T}} is the exclude group for node $x$ on tree $T$. It is the 
        set complement of \leafLabels{x} with respect to the 
    \notitem{\displaysPred{T}{x}} is an identity function that evaluates to 1 if tree $T$
        displays $x$, and to 0 otherwise.
\end{compactenum}


\bibliography{otcetera}
\end{document}
