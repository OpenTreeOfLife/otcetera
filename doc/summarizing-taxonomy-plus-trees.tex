\documentclass[11pt]{article}
\usepackage{amssymb}
\usepackage{amsmath}
\usepackage[authoryear,round]{natbib}
\bibliographystyle{plainnat}
\usepackage{bm}
\usepackage{color}
\usepackage{graphicx}
\newtheorem{theorem}{Theorem}
\newtheorem{lemma}[theorem]{Lemma}
\newtheorem{corollary}[theorem]{Corollary}
\newtheorem{definition}{Definition}
\usepackage{xspace}
\usepackage{paralist}
\usepackage{setspace}
\textwidth 6.5in
\hoffset -1in
\singlespacing

% Abbreviations
\newcommand{\otol}{Open Tree of Life\xspace}
\newcommand{\ps}{phylogenetic statement\xspace}
\newcommand{\pss}{phylogenetic statements\xspace}
\newcommand{\PSs}{Phylogenetic Statements\xspace}
\newcommand{\PS}{Phylogenetic Statement\xspace}
\newcommand{\SWIPSD}{Sum of Weighted Input \PSs Displayed\xspace}
\newcommand{\MSWIPSD}{Maximum \SWIPSD \xspace}
\newcommand{\newick}[1]{\texttt{#1}\xspace}
\newcommand{\comment}[1]{{\color{red} \textsc{#1}}\xspace}
\newcommand{\NeedsAlgorithmicWork}{{\comment{This needs algorithmic work.}}}
% Notation
\newcommand{\leafLabels}[1]{\ensuremath{\mathcal{L}(#1)}}
\newcommand{\parent}[1]{\ensuremath{\mbox{par}(#1)}}
\newcommand{\children}[1]{\ensuremath{\mbox{children}(#1)}}
\newcommand{\taxonomy}[0]{\ensuremath{\mathbb{T}}}
% verbatim, verbatim notation for a \ps
\newcommand{\vvps}[2]{\ensuremath{{#1}\downarrow{#2}}}
% leaf set, verbatim notation for a \ps
\newcommand{\lvps}[2]{\ensuremath{\leafLabels{#1}\uparrow{#2}}}
\newcommand{\leafComp}[2]{\ensuremath{\widetilde{\mathcal{L}_{#2}}\left({#1}\right)}}
\newcommand{\displaysPred}[2]{\ensuremath{\mathbb{I}_d(#1, #2)}}

\DeclareMathOperator*{\argmax}{\arg\!\max}

\usepackage{hyperref}
\hypersetup{backref,  linkcolor=blue, citecolor=black, colorlinks=true, hyperindex=true}
\begin{document}
The source for this in the doc subdirectory of the otcetera
    repo \url{https://github.com/OpenTreeOfLife/otcetera/tree/doc/doc}.
\begin{center}
    {\bf Summarizing a taxonomy and multiple estimates of phylogenetic trees} \\
{Mark T.~Holder$^{1,2,\ast}$. feel free to contribute and add your name}
\end{center}
\tableofcontents
\section{Background}
The \otol project is attempting to build a platform for summarizing what is known
    about phylogenetic relationships across all of Life.
Presenting an easy-to-interpret summary of trees that have been ``curated''
    is one component of that effort.
The project has decided that this summary should include a tree which
\begin{compactenum}
    \item can be served and browsed;
    \item contains annotations indicating which input trees support a particular grouping;
    \item has all of the tips of the taxonomy;
    \item displays as many of the groups in the input trees as is feasible;
    \item may utilize ranking of trees;
    \item (tentative - not sure if everyone is on board with this one) has no unsupported groups;\label{noUnsupportedReq}
    \item (tentative) does not ``prefer'' lack of resolution (defined more thoroughly below)
\end{compactenum}
This is results in a new form of a supertree problem described below as the ``taxonomy-based tree summary'' problem.
\subsection{Taxonomy-based supertree}
This is a novel name for a special form of the supertree problem.
A taxonomy-base supertree has at least one input (the taxonomic tree) is complete.
\subsubsection{Taxonomy-based summary tree}
Many supertree approaches seek to maximize accuracy according to some notion
    of distance between the true tree and estimated true.
The summary that we seek has requirements (e.g. \ref{noUnsupportedReq})
    which lead to less resolved trees.
Such trees may fail to display all of the well-supported (or even uncontested)
    rooted triples, but such trees also make it easier for users to see the
    connection between the summary tree and the input trees.
Thus, the phrase ``Taxonomy-based summary tree'' is used here to describe a
    taxonomy-based supertree which tries to maximize some notion of
    explaining the set of input trees (rather than a supertree designed to
    display the highest number of groups in the inputs, or some other criterion).

\subsubsection{Taxonomy-based summary of ranked input trees}
We have been pursuing a strategy that uses a ranking of trees.
In cases of conflict, the grouping that is compatible with the 
    higher ranked tree is shown in the tree (if it is not contradicted by 
    another grouping from trees of even higher rank).
 The taxonomic tree is considered to be the lowest ranked input.

Applying a ranking to the input trees is biologically questionable (because
    the importance of an input \ps should be based on the support for that
    grouping -- it is rarely the case that tree-wide rank would adequately
    describe the degree of statistical support for that group).
Using tree-based ranks also introduces subjectivity into the summary buildind process.

Nevertheless, tree-rank based summaries represent a reasonable starting point
    because they are easy to explain and the ranking permits some algorithmic
    simplifications.

\subsection{The Sum of Weighted Input \PSs Displayed Score}
Let $\mathcal{P}$ be a multiset of \pss.
If each member, $i$, this set is assigned a weight, $w_i$, then the 
sum of weighted input \pss displayed score, $S_w$, for a tree $T$ is:
\begin{equation}
    S_w(T, \mathcal{P}) = \sum_{i\in \mathcal{P}} w_i \displaysPred{T}{i}
\end{equation}
where {\displaysPred{T}{x}} is an identity function that evaluates to 1 if tree $T$
        displays $x$, and to 0 otherwise.

\subsubsection{\MSWIPSD problem}
Trying to find the set of trees that maximize $S_w$ is one natural
    goal for a summarization procedure.
We can call this the ``\MSWIPSD''
problem,
    and the set of trees that maximize the score are denoted:
\begin{equation}
    \mathcal{S}_{w}(\mathcal{P}) = \argmax_T S_w(T,\mathcal{P}) 
\end{equation}

Tree-based ranking can be viewed as a means providing weights to \pss.
Each tree is converted to a set of \pss, and the tree's weight is 
    assigned to each \ps in the set.
The union of the tree's sets becomes the multiset, $\mathcal{P}$, referred to
    in the definition of the score above.

\href{https://github.com/OpenTreeOfLife/treemachine/wiki/MaxWeightOfInputTreeEdgesDisplayed}{This link}
    sketches out a proof of how an extreme form of tree-based ranks
    can lead to greedy tree addition strategy can be guaranteed to 
    find the set of trees that solve the \MSWIPSD problem.
If the difference in tree ranks are sufficiently large from one tree to the next,
    then there is no need to consider skipping a \ps from a high ranking tree
    even if not considering that grouping would result in all of the \ps
    from the lower ranking trees being displayed.

Unfortunately, that proof only applies to an exact algorithm which returns
    all possible trees that maximize the score.
We do not know of a polynomial-time algorithm for solving that problem.

\subsection{Trees without unsupported groups}
Consider an edge connecting parent $\parent{V}$ to it child $V$ in a complete tree $T$.
This edge is supported (or ``the node $V$ is supported'') in the sense of the \MSWIPSD score if
    collapsing the edge leads to a lower (worse) score.
For an edge to be supported in this sense, it must display in input \pss, and
    the tree that woudld be created by collapsing the edge to a polytomy must
    {\em not} display the \pss.

Equivalently, we can state the conditions node $V$ (or it subtending) begin supported
    by an input \pss derived from node $x$ of tree $t$ as:
\begin{eqnarray}
    \leafLabels{V} \cap \leafLabels{t} & = & \leafLabels{x}\\
    \left(\leafLabels{\parent{V}} \cap \leafLabels{t}\right) - \leafLabels{x} & \neq &\emptyset\\
    \leafLabels{c} \cap \leafLabels{t} & \neq & \leafLabels{x} \hskip 2em \forall c \in \children{V}
\end{eqnarray}
where $\children{V}$ is the set of nodes that are children of $V$.
The first condition guarantees that the node $V$ displays the \ps made by $x$.
The second condition assures that if the edge leading to $V$ were collapsed, the resulting polytomy would not display the \ps.
The final condition assures that none of the children of $V$ also display this \ps from $x$;
    if any child displayed the \ps, then collapsing the edge leading to $V$ would still
    yield a tree that still displays the \ps derived from $x$.

If we demand that a summary tree contains no unsupported groups (in the previous sense of support), we 
    are constraining the set of summary trees such that, for every internal node in the summary tree
    there is at least one \pss in the input set that supports it.
This restricts the number of edges in the tree: as the sum of the number of internal nodes in the inputs becomes an upper bound
    on the number of internal nodes in the summary tree.
However, restricting the summary to only contain supported groups does not decrease the 
    number of \pss from the inputs that are displayed.
Furthermore, any maximal scoring solution can be converted to a maximal scoring
    solution by collapsing edges one at a time and checking for the existence of further
    unsupported groups.
If $\mathcal{S}_s$ is the subset of $\mathcal{S}_w$ which do not contain any unsupported nodes,
    then $\mathcat{S}_s$ is never empty, but may be much smaller than $\mathcal{S}_w$ because
    every resolution of $\mathcal{S}_s$ is a member of $\mathcal{S}_w$.

\subsection{Evaluating how well a single summary tree summarizes a set of summary trees}\label{treeAdmissibility}
As discussed above, we might try to seek the set of trees that contain no unsupported nodes and that
    maximize the \SWIPSD score.
However, one of the requirements is that we return a single tree.

One can interpret an unresolved tree as a set of trees - specifically the set of trees that 
    can be produced by resolving the tree.
We may be able to formalize a score for a single summary trees, $T$, by considering
    the set of trees that can be produced by resolving it, calling this set $\mathcal{R}(T)$.
Specifically, if our primary summary is a set of trees $\mathcal{S}$, we may want
    to evaluate $T$ by the number of trees in $\mathcal{S}$ which are not found
    in $\mathcal{R}(T)$ 
    and the number (or proportion) of trees in $\mathcal{R}(T)$ which
    are not in $\mathcal{S}$.
Both of these statistics would be small if $T$ is a good summary of $\mathcal{S}$; they 
    would be zero if the set of trees to be summarized is identical to the resolutions
    of the tree.
Note this pair of sets (the ``false negative'' and ``false positive'' sets) are the
    sets that are calculate in when calculating the symmetric difference between 
    sets (and related statistics such as the Robinson-Foulds distance in phylogenetics).

Given the difficulty enumerating either $\mathcal{S}$ or $\mathcal{R}(T)$, we may not be able to easily 
    apply this method of scoring trees often.
It would also be difficult to figure out an appropriate weighting of false positives vs false negatives.

However, there may be cases in which we compare two very similar trees, $A$ and $B$, it is obvious
    that $A$ has a lower value than $B$ for one statistic and an equal-or-lower value for the other.
Using an analogy to statistics, we would say that $A$ dominates $B$ and that $B$ is an inadmissible summary.

\subsection{Relationship between a series of trees and a series \pss}
In terms of the \MSWIPSD problem, the set of input trees can converted to a multiset of input \pss without 
    altering the solution because no aspect of the scoring system depends on whether the input \ps
    which are displayed were derived from the same tree.

As mentioned above, using a ranking system that very strongly favors the more highly ranked trees
    simplifies the search for a solution to the \MSWIPSD problem.
Thus a ranked list of trees (highest priority to lowest) can be mapped to a list of sets of \pss.
A greedy approach that tries builds up a solution by adding one \ps at a time could generate the optimal
    set of summary trees could be guaranteed to work if the input order is correct.
The greedy solver would have to accept as many splits as possible, and avoid rejecting a \ps unnecessarily,
    but it would be greedy in the sense that it does not have to ``look ahead'' or reconsider a split
    that it has accepted or rejected.

Unfortunately, it is not clear how to convert the ranked list of trees to a ordering of the splits.
The ranked list of sets of \pss that can be naturally derived from the rankes list of trees only provides
    a partial order.

I need to dig through my notes, but I think that there are cases for which the order of adding trees within
    a tree affects the output.

Checking all possible input orders would be one solution. 

Currently, otcetera and peyotl-based supertree steps just use a postorder traversal (which is arbitrary with
    respect to the order of sister groups). \NeedsAlgorithmicWork

\subsection{The interpretation of input trees with tips mapped to non-terminal taxa}
Some of the input phylogenetic estimates may have leaves that are not mapped to terminal taxa.
The correct biological interpretation of such labels is not clear.

Some possible meanings of a leaf in a tree being mapped to a non-terminal taxon, $A$:
\begin{compactenum}
    \item $A$ should be a terminal taxon - the reference taxonomy is incorrect.
    \item the taxon $A$ is asserted or assumed by the authors of the study to be monophyletic.\label{itmMonophyleticTip}
    \item at least one descendant taxon of $A$ occurs at this point in the tree, but it is not know which descendant.\label{itmUnknownTip}
    \item the phylogenetic analysis was conducted using a ``chimeric'' set of character data drawn from multiple
        members of the taxon $A$.
    \item a non-extant lineage was sampled and included in the phylogenetic analysis. That tip is thought to be:
    \begin{compactenum}
        \item the most recent common ancestor of taxon $A$,
        \item an anestor of taxa that are descendants of $A$ (but we don't know which one), OR
        \item an extinct taxon that is a member of $A$.
    \end{compactenum}
\end{compactenum}
Presumably case \ref{itmUnknownTip} is the most common case in our corpus.
Even if case \ref{itmMonophyleticTip} is the case for some trees+leaf combinations, I presume
    that we would want the source of such phylogenetic claims to be more transparent.
Thus, I assume that we do {\em not} want such tips to be interpreted as providing evidence
    for monophyly of the non-terminal tip.

\subsubsection{expanding non-terminals to the contained terminals}
If the taxon is monophyletic based on other trees in the corpus, these cases are not too not problematic.
One could simply transform the tip to a polytomy containing all of the terminal taxa that are
    descendants of the mapped taxon as children of the polytomy.
This input representation would imply that the input tree supported the monophyly of the taxon, but 
    that could be rectified by making note of the fact that the polytomy was an expansion of a tip
    and later suppressing any annotation that claims that the tree supports monophyly.

One could also follow this expansion procedure in the case of contested taxa.
However, that would presumable entail more care to ensue that the \ps that corresponds
    to the polytomy does not contribute to the topological decisions during the 
    tree construction.

\subsubsection{expanding non-terminals to the contained terminals attached to the parent of the leaf}
As described above, expanding the non-terminals to their contained terminals requires 
    some bookkeeping to note that the internal node produced should not generate a \ps that
    is taken to be an input.
If we are calculated ``supported by'' statements about the summary tree as a post-processing step (rather
    than propagating that information at every step of the pipeline), it is sufficient
    to transform the input tree with a non-terminal taxon mapping to tree that does not
    claim monophyly non-terminal taxon.
This can be done by creating the polytomy of terminal leaves at the parent node of the node that is
    mapped to a non-terminal taxon (and pruning the ``barren'' leaf that is the remnant of the tip
    mapped to the non-terminal taxon).

\subsubsection{optimizing the assignment to a terminal taxon}
Under the unknown tip case (case \ref{itmUnknownTip} above), we could treat the
    correct assignment of the non-terminal tip to a terminal tip as an
    unknown, latent variable to be optimized.
In other words, we would try to make the assignments in such a way as to
    maximize the \SWIPSD score.
This sounds like it would lead to a combinatorial explosion in complexity when 
    we have multiple trees that use the same non-terminal taxon in the corpus.
So, as far as I know, we have not seriously considered this.

\subsubsection{pruning non-terminal taxon tips}
We have many tips that are pruned from the input trees because they are 
    not correctly mapped to a taxon in the reference taxonomy.
We could adopt the unknown tip (case \ref{itmUnknownTip} above) interpretation,
    and prune these tips.
This seems a bit draconian and wasteful - particularly given the the study
    curation tool does not warn about non-terminal mapping.

\subsubsection{pruning non-terminal taxon tips if the terminal taxon is not monophyletic}
We might view leaves mapped to non-terminal taxa as hopelessly ambiguous Whenever the non-terminal
    taxon is not monophyletic (based on other trees).
Thus, we could prune these cases when other trees reject monophyly.

This seems more reasonable that unconditionally pruning them, but more difficult to implement.
A higher ranked \ps might contest the monophyly of the taxon, but that \pss might be 
    in conflict with even more highly ranked \pss.
There may be some clever trick for determining whether a taxon will be monophyletic in the
    final tree without performing synthesis iteratively.

\subsubsection{pruning non-terminal taxon tips if the terminal taxon is contested}
This is a proposal that is intermediate between the previous 2 proposals.
It is easy to test for whether or not a taxon is contested.
However, if their are high ranking \pss that supprot the monophyly of the taxon,
    then this procedure may prune tips that are not really ambiguous given 
    the full data.

\subsection{Proposed formalization of the goal}
It would be great if the summary draf tree would be:
    an admissible summary tree ({\em sensu} section \ref{treeAdmissibility}) of the set of trees
    that maximize the ranked-tree \SWIPSD score.

Unfortunately that is probably infeasible.  I think that a reasonable back up would be to 
    produce a summary tree which:
\begin{compactenum}
    \item displays every uncontested taxon, and
    \item shows an admissable summary of the \MSWIPSD set for each subproblem that is
       created by tiling the tree into the contested subproblems.
\end{compactenum}
\subsection{Decomposition into uncontested taxon subproblems}
One can efficiently:
\begin{compactenum}
    \item determine whether a taxon context is contested,
    \item resolve any polytomy in an input tree which can be resolved by constraining
        the set of uncontested taxa, and
    \item produce a subproblem or each uncontested taxon. Each subproblem just contains
        a subset of each input tree that overlaps with this subproblem
\end{compactenum}

MTH conjectures that constraining uncontested taxa will not force the \SWIPSD score to decrease.

Note that in general, enforcing the presence of a contested \pss may increase the 
    \SWIPSD score.
For example, consider the ranked inputs:
\begin{compactenum}
\item \newick{((A,B),C)}
\item \newick{((A,C),D)}
\item \newick{((A,D),B)}
\end{compactenum}
The third tree displays a grouping that is not contested by either of the other 2 groups, but
    if we force that grouping to be present in the full tree, that full tree cannot 
    display both of the higher ranked \pss.

MTH's conjecture is: If a complete \pss is uncontested by any input, then there exists 
a summary tree which both displays the \ps and attains the highest possible \SWIPSD score.


\subsubsection{implementation notes: otc-uncontested-decompose}
This operation is implement in the {\tt otc-uncontested-decompose}.
It takes a taxonomic tree and each of the input trees (in ranked order), and a flag
    that specifies which output directory should hold the subproblems.
It uses an embedding approach
The subproblem is simply a induced
        tree for the resolved input trees.
        The nodes used to induce the subtree are the set of 

\section{Pipeline}


\newcommand{\defitem}[2]{\item[{\bf #1}]\label{itm:#2} }
\newcommand{\notitem}[1]{\item[]#1}
\section{Glossary}
\begin{compactenum}
\defitem{as complete}{defAsComplete} a phrase contrasting 2 trees.
    Tree $A$ is as complete as tree $B$ $\iff$
    leaf label set of $B$ is a subset of the leaf label set of $A$.\\
    In notation: $\leafLabels{B} \subseteq \leafLabels{A}$.
\defitem{cluster}{defCluster} the set of leaf nodes that are descendants
    of a node, $x$, is the cluster of $x$
\defitem{complete}{defComplete} as an adjective for a tree. Tree $A$ is complete $\iff$ the leaf label set
    of tree $A$ is identical to the the set of terminal taxa.
    In notation: $\leafLabels{A} = \leafLabels{\taxonomy}$
\defitem{conflicting}{defConflicting}. 2 \pss conflict with each other if there does not exist
    any tree that displays both of them.\\
    Since we are dealing with rooted statements that can partially overlap in term of their leaf sets, 
    a procedure for checking for conflict: restrict both statements to just the overlapping labels 
    (remove any labels that are not in the intersection of the leaf sets of the two statements), the
    \pss are in conflict if their include groups overlap but neither include group is a subset of the
    other.
\defitem{contested}{defContested} as an adjective for a non-terminal taxon. Taxon $A$ is contested $\iff$ 
    there is at least one tree in the set of input trees that has a node which is in conflict
    with the taxon.
    Note that if a tree has members of taxon as children of a polytomy that also contains other taxa, then
        the tree does not display the taxon, but it is also does not contest the taxon.
    Thus ``contests'' is not equivalent to ''does not display'' (though, if a tree displays a taxon, then 
        it cannot contest that taxon).
\defitem{display}{defDisplay} Tree $A$ displays a node, $x$, from tree $B$ $\iff$
    that $A$ is as complete as tree $B$ and the induced tree of $A$ limited to the 
    leaf label set of $B$ contains a cluster with labels identical to the label set of the cluster of $x$.\\
    An equivalent characterization in terms of the full leaf set of $A$ would be:
    tree $A$ is as complete as $B$ and there is a node in $A$ which has a leaf label set 
    that is a superset of \leafLabels{x} and which does not contain any member of $\leafLabels{B} \setminus \leafLabels{x}$.\\
    We say that a tree $A$ displays a \ps $x$, if you were to translate
        $x$ tree with single internal node $y$, and tree $A$ displays $y$.
\defitem{internal node}{defInternalNode} a node that is not the root or a leaf.
\defitem{label}{defLabel} When speaking of the label of a node in this document, we are referring to unique taxonomic
    identifier for that node.
    This document does not discuss any issues associated with mapping strings to taxonomic names.
\defitem{leaf labels}{defLeafLabels} For our problems, the input trees have been aligned to a common
    taxonomy. 
    So referring to the leaf label set of a tree can be interpreted as
    ``the set of taxonomic identifiers that are mapped to the leaves of the tree''.
    The leaf label set of a node is the set of labels associated with the cluster of the node.\\
    Note that a leaf label is not necessarily a terminal taxon.
    In notation, $\leafLabels{x}$ is the leaf label set of $x$.
\defitem{more complete}{defMoreComplete} tree $A$ is {\em more complete} than tree $B$ $\iff$ the leaf
    label set of $B$ is a proper subset of the leaf label set of $A$. \\
    In notation, $\leafLabels{B} \subset \leafLabels{A}$.
\defitem{phylogenetic statement}{defPS} (we are considering ``rooted split'' or ``rooted bipartition'' for this phrase)
    This is a pair of sets of taxonomic id sets: the include set and the exclude set (called ``ingroup'' and ``outgroup''
    on the google doc).\\
    The interpretation of a \ps is: if the statement is true, then 
    any pair of elements in the include are more closely related to each
        other than they are to any element in the the exclude set.
    Thus if there are $N$ labels in the include set, and $M$ in the exclude set, the \ps implies
    ${N \choose 2}M$ distinct rooted triples.\\
    The include set and the exclude set must be disjunct. Their union is referred to as the leaf label set
        for the \ps.
    When we use the phrase ``\ps'' without qualifier, we mean a nontrivial statement.
    For a \ps to be nontrivial there must be at least 2 elements in the include group and at least 
        1 in the exclude group.\\
    We use ``\vvps{\mbox{include}}{\mbox{exclude}}'' to denote a \ps. 
    This is very similar to split syntax in unrooted phylogenetics, but with the arrow 
        in place of $\mid$ to emphasize that the statement is oriented.\\
    Each internal node in a tree maps to a nontrivial \ps that can created by setting
        the include set equal to the leaf labels of node and setting the exclude set equal
        to the leaf label set of the tree minus the leaf labels of the node.
        In notation, node $x$ in tree $T$, maps to: $\lvps{x}{[\leafLabels{T} \setminus \leafLabels{x}]}$.\\
    A tree can be converted to a set of \pss: one for each internal node in the tree.
\defitem{terminal taxon}{defTerminalTaxon} a taxonomic identifier for at taxonomic node that
    has no children {\em in the taxonomic tree}.
\defitem{trivial}{defTrivial} a trivial \ps is one which must be found and any tree that has
    the leaf labels set of the \ps.
    This includes

\end{compactenum}
\section{Notation}
\begin{compactenum}
    \notitem{\leafLabels{x}} is the leaf label set of $x$.
    \notitem{\taxonomy} the taxonomic tree.
    %\notitem{\leafComp{x}{T}} is the exclude group for node $x$ on tree $T$. It is the 
        %set complement of \leafLabels{x} with respect to the 
    \notitem{\displaysPred{T}{x}} is an identity function that evaluates to 1 if tree $T$
        displays $x$, and to 0 otherwise.
\end{compactenum}


\bibliography{otcetera}
\end{document}
