\documentclass[11pt]{article}
\usepackage{amssymb}
\usepackage{amsmath}
\usepackage{pdfpages}
\usepackage{graphicx}
\usepackage[authoryear,round]{natbib}
\usepackage{algorithm,algorithmic}
\bibliographystyle{plainnat}
\usepackage{color}
\usepackage{graphicx}
\newtheorem{theorem}{Theorem}
\newtheorem{lemma}[theorem]{Lemma}
\newtheorem{corollary}[theorem]{Corollary}
\newtheorem{conjecture}[theorem]{Conjecture}
\newtheorem{definition}{Definition}
\usepackage{xspace}
\usepackage{paralist}
\usepackage{setspace}
\textwidth 6.5in
\textheight 10in
\hoffset -1in
\voffset -1.4in
\singlespacing
\parindent 0em
\parskip .4em
% Abbreviations
\newcommand{\otol}{Open Tree of Life\xspace}
\newcommand{\ps}{phylogenetic statement\xspace}
\newcommand{\pss}{phylogenetic statements\xspace}
\newcommand{\PSs}{Phylogenetic Statements\xspace}
\newcommand{\PS}{Phylogenetic Statement\xspace}
\newcommand{\SWIPSD}{Sum of Weighted Input \PSs Displayed\xspace}
\newcommand{\MSWIPSD}{Maximum \SWIPSD \xspace}
\newcommand{\newick}[1]{\texttt{#1}\xspace}
\newcommand{\otc}[0]{\texttt{otcetera}\xspace}
\newcommand{\otcprune}[0]{\texttt{otc-prune-taxonomy}\xspace}
\newcommand{\otcdecompose}[0]{\texttt{otc-uncontested-decompose}\xspace}
\newcommand{\nexson}[0]{\texttt{otNexSON}\xspace}
\newcommand{\gcmdr}[0]{\texttt{gcmdr}\xspace}
\newcommand{\simplification}[0]{\\\noindent\textsc{Simplification Action(s)}:\xspace}
\newcommand{\undoActions}[0]{\\\noindent\textsc{``Undo'' Simplification Action(s)}:\xspace}
\newcommand{\stepExplanation}[0]{\\\noindent\textsc{Explanation}:\xspace}
\newcommand{\stepInput}[0]{\\\noindent\textsc{Input}:\xspace}
\newcommand{\stepOutput}[0]{\\\noindent\textsc{Output}:\xspace}
\newcommand{\currImpl}[0]{\\\noindent\textsc{Current Impl.}:\xspace}
\newcommand{\implTODO}[0]{\\\noindent\textsc{TODO for Impl.}:\xspace}
\newcommand{\currURL}[0]{\\\noindent\textsc{URL for output}:\xspace}
\newcommand{\comment}[1]{{\color{red} \textsc{#1}}\xspace}
\newcommand{\TODO}[1]{\comment{TODO: #1}}
\newcommand{\NeedsAlgorithmicWork}{{\comment{This needs algorithmic work.}}}
\newcommand{\ProofWriteupNeeded}{{\comment{Need to write up this proof}}}

\newcommand{\incLSSS}{Include-LeafSet support statement\xspace}
\newcommand{\incLSSSs}{Include-LeafSet support statements\xspace}

% Notation
\newcommand{\pssInOptimalTree}{\ensuremath{\hat{\mathcal{G}}}\xspace}
\newcommand{\pssFrom}[1]{\ensuremath{\mathcal{G}(#1)}\xspace}
\newcommand{\tripleSetInOptimal}{\ensuremath{\hat{\mathcal{R}}}\xspace}
\newcommand{\leafLabels}[1]{\ensuremath{\mathcal{L}(#1)}}
\newcommand{\parent}[1]{\mbox{parent}(#1)}
\newcommand{\children}[1]{\mbox{children}(#1)}
\newcommand{\nodes}[1]{\mbox{nodes}(#1)}
\newcommand{\treeRoot}[1]{\mbox{root}(#1)}
\newcommand{\taxonomy}[0]{\ensuremath{\mathbb{T}}\xspace}
\newcommand{\prunedTaxonomy}[0]{\ensuremath{\mathbb{T}_P}\xspace}
\newcommand{\phyloInputs}[0]{\ensuremath{\mathcal{T}}}
\newcommand{\expandedPhylo}[0]{\ensuremath{\mathcal{T}_{E}}\xspace}
\newcommand{\prunedSummary}[0]{\ensuremath{\mathcal{S}_{P}}\xspace}
\newcommand{\summaryTree}[0]{\ensuremath{\mathcal{S}}\xspace}
% verbatim, verbatim notation for a \ps
\newcommand{\vvps}[2]{\ensuremath{{#1}\downarrow{#2}}}
% leaf set, verbatim notation for a \ps
\newcommand{\lvps}[2]{\ensuremath{\leafLabels{#1}\downarrow{#2}}}
\newcommand{\leafComp}[2]{\ensuremath{\widetilde{\mathcal{L}_{#2}}\left({#1}\right)}}
\newcommand{\displaysPred}[2]{\ensuremath{\mathbb{I}_d(#1, #2)}}

\newcommand{\leafDes}[1]{\ensuremath{\mathcal{L}_d(#1)}}
\newcommand{\excLeafDes}[1]{\ensuremath{\mathcal{L}_e(#1)}}
%\newcommand{\children}[1]{\ensuremath{\mathcal{C}(#1)}}

\newcommand{\mrca}[2]{\ensuremath{\mbox{mrca}(#1, #2)}}
\newcommand{\treeOf}[1]{\ensuremath{\mbox{tree}(#1)}}
\newcommand{\leafSet}[1]{\leafLabels{#1}}}
\newcommand{\cLeafSet}[1]{\texttt{{#1}.leafLSet}}}
\newcommand{\cDes}[1]{\texttt{{#1}.desIds}}}

\newcommand{\supportingPSSetFull}[2]{\ensuremath{\mbox{\texttt{SuppStatementSet}}[#1, #2]}}
\newcommand{\supportingPSSet}[1]{\ensuremath{\mbox{\texttt{SSS}}[#1]}}
\DeclareMathOperator*{\argmax}{\arg\!\max}


\newcommand{\phylogeny}{\ensuremath{P}\xspace}
\newcommand{\uncontestedTaxa}{\ensuremath{\mathcal{U}}\xspace}
\usepackage{hyperref}
\hypersetup{backref,  linkcolor=blue, citecolor=black, colorlinks=true, hyperindex=true}
\begin{document}
The source for this in the doc subdirectory of the otcetera
    repo \url{https://github.com/mtholder/otcetera/tree/master/doc}.
\begin{center}
    {\bf Detecting contested taxa using embedded trees} \\
{Benjamin D. Redelings$^{1}$ and Mark T.~Holder$^{1,2\ast}$}
\end{center}
\tableofcontents
\section{Contested taxa}
This document is intended to be a more thorough discussion of the
    ``Decomposition into Subproblems'' section of the Supporting
    Information of \citep{HinchliffEtAl2015}.
Here we consider the case of a complete tree representing a taxonomy \taxonomy and
    a list of $N$ phylogenetic estimates, $[P_1, P_2, \ldots, P_N]$, in which the leaf labels of the phylogenetic
    trees is a subset of the leaf label set of the taxonomy.
Note that in \citep{HinchliffEtAl2015}, some phylogenetic trees could have leaves labeled by
    ``higher'' taxa, but that possibility is not covered in this document, because in
    the \propinquity pipeline, these tips are replaced by exemplars of the higher taxon
    at a previous step.

As state in the glossary for \otc, we say that taxon $A$ is contested $\iff$ 
    there is at least one tree in the set of input trees that has a node which is in conflict
    with the taxon.
    Note that if a tree has members of taxon as children of a polytomy that also contains other taxa, then
        the tree does not display the taxon, but it is also does not contest the taxon.
    Thus ``contests'' is not equivalent to ``does not display'' (though, if a tree displays a taxon, then 
        it cannot contest that taxon).

\section{Subproblems}
Let $\uncontestedTaxa$ denote the set of higher taxa in \taxonomy.
Any labels of internal nodes in a tree that is one of the input phylogenetic estimates
    are ignored\footnoted{In the software, we actually use a syntatic structure to the
    internal node labels, so that we can associate them with specific nodes in the input tree
    but we can think of the label, for the sake of the algorithm, as being a taxonomic label
    which is removed on input.}.
Let $\mathcal{A}_i$ denote tree obtained by starting with $P_i$ and labeling internal 
    nodes or introducing new, labeled internal nodes (with in-degree and outdegree of 1)
    such that the label of each uncontested node represented in $P_i$ is attached to a node
    in $\mathcal{A}_i$.

\bibliography{otcetera}
\end{document}
