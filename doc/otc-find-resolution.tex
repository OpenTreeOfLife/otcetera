\documentclass[11pt]{article}
\usepackage{amssymb}
\usepackage{amsmath}
\usepackage{pdfpages}
\usepackage{graphicx}
\usepackage[authoryear,round]{natbib}
\usepackage{algorithm,algorithmic}
\bibliographystyle{plainnat}
\usepackage{color}
\usepackage{graphicx}
\newtheorem{theorem}{Theorem}
\newtheorem{lemma}[theorem]{Lemma}
\newtheorem{corollary}[theorem]{Corollary}
\newtheorem{conjecture}[theorem]{Conjecture}
\newtheorem{definition}{Definition}
\usepackage{xspace}
\usepackage{paralist}
\usepackage{setspace}
\textwidth 6.5in
\textheight 10in
\hoffset -1in
\voffset -1.4in
\singlespacing
\parindent 0em
\parskip .4em
% Abbreviations
\newcommand{\otol}{Open Tree of Life\xspace}
\newcommand{\ps}{phylogenetic statement\xspace}
\newcommand{\pss}{phylogenetic statements\xspace}
\newcommand{\PSs}{Phylogenetic Statements\xspace}
\newcommand{\PS}{Phylogenetic Statement\xspace}
\newcommand{\SWIPSD}{Sum of Weighted Input \PSs Displayed\xspace}
\newcommand{\MSWIPSD}{Maximum \SWIPSD \xspace}
\newcommand{\newick}[1]{\texttt{#1}\xspace}
\newcommand{\otc}[0]{\texttt{otcetera}\xspace}
\newcommand{\otcprune}[0]{\texttt{otc-prune-taxonomy}\xspace}
\newcommand{\otcdecompose}[0]{\texttt{otc-uncontested-decompose}\xspace}
\newcommand{\nexson}[0]{\texttt{otNexSON}\xspace}
\newcommand{\gcmdr}[0]{\texttt{gcmdr}\xspace}
\newcommand{\simplification}[0]{\\\noindent\textsc{Simplification Action(s)}:\xspace}
\newcommand{\undoActions}[0]{\\\noindent\textsc{``Undo'' Simplification Action(s)}:\xspace}
\newcommand{\stepExplanation}[0]{\\\noindent\textsc{Explanation}:\xspace}
\newcommand{\stepInput}[0]{\\\noindent\textsc{Input}:\xspace}
\newcommand{\stepOutput}[0]{\\\noindent\textsc{Output}:\xspace}
\newcommand{\currImpl}[0]{\\\noindent\textsc{Current Impl.}:\xspace}
\newcommand{\implTODO}[0]{\\\noindent\textsc{TODO for Impl.}:\xspace}
\newcommand{\currURL}[0]{\\\noindent\textsc{URL for output}:\xspace}
\newcommand{\comment}[1]{{\color{red} \textsc{#1}}\xspace}
\newcommand{\TODO}[1]{\comment{TODO: #1}}
\newcommand{\NeedsAlgorithmicWork}{{\comment{This needs algorithmic work.}}}
\newcommand{\ProofWriteupNeeded}{{\comment{Need to write up this proof}}}

\newcommand{\incLSSS}{Include-LeafSet support statement\xspace}
\newcommand{\incLSSSs}{Include-LeafSet support statements\xspace}

% Notation
\newcommand{\pssInOptimalTree}{\ensuremath{\hat{\mathcal{G}}}\xspace}
\newcommand{\pssFrom}[1]{\ensuremath{\mathcal{G}(#1)}\xspace}
\newcommand{\tripleSetInOptimal}{\ensuremath{\hat{\mathcal{R}}}\xspace}
\newcommand{\leafLabels}[1]{\ensuremath{\mathcal{L}(#1)}}
\newcommand{\parent}[1]{\mbox{parent}(#1)}
\newcommand{\children}[1]{\mbox{children}(#1)}
\newcommand{\nodes}[1]{\mbox{nodes}(#1)}
\newcommand{\treeRoot}[1]{\mbox{root}(#1)}
\newcommand{\taxonomy}[0]{\ensuremath{\mathbb{T}}\xspace}
\newcommand{\prunedTaxonomy}[0]{\ensuremath{\mathbb{T}_P}\xspace}
\newcommand{\phyloInputs}[0]{\ensuremath{\mathcal{T}}}
\newcommand{\expandedPhylo}[0]{\ensuremath{\mathcal{T}_{E}}\xspace}
\newcommand{\prunedSummary}[0]{\ensuremath{\mathcal{S}_{P}}\xspace}
\newcommand{\summaryTree}[0]{\ensuremath{\mathcal{S}}\xspace}
% verbatim, verbatim notation for a \ps
\newcommand{\vvps}[2]{\ensuremath{{#1}\downarrow{#2}}}
% leaf set, verbatim notation for a \ps
\newcommand{\lvps}[2]{\ensuremath{\leafLabels{#1}\downarrow{#2}}}
\newcommand{\leafComp}[2]{\ensuremath{\widetilde{\mathcal{L}_{#2}}\left({#1}\right)}}
\newcommand{\displaysPred}[2]{\ensuremath{\mathbb{I}_d(#1, #2)}}

\newcommand{\leafDes}[1]{\ensuremath{\mathcal{L}_d(#1)}}
\newcommand{\excLeafDes}[1]{\ensuremath{\mathcal{L}_e(#1)}}
%\newcommand{\children}[1]{\ensuremath{\mathcal{C}(#1)}}

\newcommand{\mrca}[2]{\ensuremath{\mbox{mrca}(#1, #2)}}
\newcommand{\treeOf}[1]{\ensuremath{\mbox{tree}(#1)}}
\newcommand{\leafSet}[1]{\leafLabels{#1}}}
\newcommand{\cLeafSet}[1]{\texttt{{#1}.leafLSet}}}
\newcommand{\cDes}[1]{\texttt{{#1}.desIds}}}

\newcommand{\supportingPSSetFull}[2]{\ensuremath{\mbox{\texttt{SuppStatementSet}}[#1, #2]}}
\newcommand{\supportingPSSet}[1]{\ensuremath{\mbox{\texttt{SSS}}[#1]}}
\DeclareMathOperator*{\argmax}{\arg\!\max}

\usepackage{xr}
\externaldocument[S-]{summarizing-taxonomy-plus-trees}
\newcommand{\summDoc}{\href{http://phylo.bio.ku.edu/ot/summarizing-taxonomy-plus-trees.pdf}{the otcetera summary}\xspace}
\newcommand{\summRef}[1]{Section \ref{S-#1} of \summDoc}
\usepackage{hyperref}
\hypersetup{backref,  linkcolor=blue, citecolor=black, colorlinks=true, hyperindex=true}
\begin{document}
\section*{\texttt{otc-find-resolution}}
This document describes the behavior of \texttt{otc-find-resolution} when the \texttt{-u}
    option is used to request resolution of tree without introducing unsupported nodes.

\subsection*{Background and notation}
Consider a supertree \summaryTree that displays a set of input \pss \pssInOptimalTree.
We say that \summaryTree is {\em minimal} with respect \pssInOptimalTree, if there is
    no edge in \summaryTree that can be collapse to result in a smaller tree $\summaryTree^{\prime}$
    which still displays all of the \pss in \pssInOptimalTree.

$\leafLabels{T}$ is the set of leaf labels associated with the tips of a tree.
Let $\leafDes{n}$ represent the set of leaf label that are associated with node that are descendants
    of node $n$.
$\excLeafDes{n, T}$ is the set of leaf labels of the tree T which are {\em not} descendants of $n$.
Thus $\leafLabels{T} = \leafDes{n} \cup \excLeafDes{n,T}$ and $\leafDes{n} \cap \excLeafDes{n,T}=\emptyset$ for any node $n$ in $T$.

$\excLeafDes{n}$ is shorthand for $\excLeafDes{n, T}$ where $T$ is the tree that $n$ belongs to.

$\pssFrom{T}$ is the set of \pss that correspond to the non-root, clusters of tree $T$.
If $T$ denotes a set or series of trees, then this notation is used to indicate the union of the \pss from each tree.

$\children{n}$ is the set of nodes that are children of $n$.
$\parent{n}$ is the parent of $n$.
$\mrca{n, T}$ is least inclusive ancestor of $\leafDes{n}$ in tree $T$.

An \incLSSS is a tuple of an include leaf set and a full leaf set from an input \pss.
The exclude group for the \pss is simply the difference between the leaf set and the include group.
This format can be more memory efficient because each \incLSSS from the same input tree
can refer to the same leaf set.

\subsection*{Input}
\begin{compactitem}
    \item A tree representing the full taxonomy, \taxonomy
    \item A summary tree, $\summaryTree$, which has no nodes of outdgree-1 and which is minimal wrt \pssInOptimalTree.
    \item A series of input phylogenetic estimates  $[T_1, T_2, \ldots T_{N-1}]$ in ranked order of importance (1 being the most important)
\end{compactitem}
The full set of inputs is taken to be: $\mathcal{T} = [T_1, T_2, \ldots T_{N}]$ where $T_N = \taxonomy$.
    $\pssInOptimalTree$ is the subset of $\pssFrom{\mathcal{T}}$ which are displayed by $\summaryTree$.

\subsection*{Goal}
Print a tree $\summaryTree^{\prime}$ which:
\begin{compactitem}
    \item is a more resolved form of $\summaryTree$.
    \item is minimal wrt $\pssInOptimalTree^{\prime}$ where $\pssInOptimalTree \subset \pssInOptimalTree^{\prime}$ and $\pssInOptimalTree^{\prime} \subset \pssFrom{\mathcal{T}}$. and 
    \item groupings are added to the tree in the order of the ranking of the trees (to maximize the \SWIPSD score)
\end{compactitem}

This is a greedy approximation that attempts to output the highest scoring resolution of $\summaryTree$ under the \SWIPSD score when the
    weights of for each tree is much greater than the weight of the next ranked tree.

\section*{Implementation}
\begin{compactenum}
    \item Read all input trees. Let $\summaryTree^{\prime}$ denote the current state of tree that is initialized by the input
    \summaryTree throughout the procedure. 
    As the tree becomes more resolved $\summaryTree^{\prime}$ will always denote the ``current'' form of the tree.
    \item Any tip $n$ in $\mathcal{T}$ mapped to a non-terminal taxon is expanded as described in \summRef{expandNonTermPar}.
    This creates of $\leafDes{n}$ attached to $\parent{n}$ and removes $n$ from the tree.
    \item The non-root internal nodes $x$ of $\summaryTree$ are each decorated with a  $\supportingPSSet{x}$ data structure.\footnote{A
    more complete name for this structure would be something like $\supportingPSSetFull{x}{\pssFrom{\mathcal{T}}}$,
    because it depends on the $\pssInOptimalTree$.
    We will call it $\supportingPSSet{x}$ for brevity, and because this set of inputs fixed for each run.}
    This data structure holds a set of  {\incLSSSs}.
    Every member of $\supportingPSSet{x}$ will support node $x$ in the
    sense that if the edge subtending $x$ were suppressed then the tree would no longer 
    display the \ps that contributed this \incLSSS.
    \item Each node $x$ in a tree $T$ is decorated with a $\texttt{desIds}$ which holds $\leafDes{x}$;
    each node also has $\texttt{leafSet}$ which refers to $\leafSet{T}$.
    \item Walk through each tree in $\mathcal{T}$ in order of rank. For each tree, 
    traverse it visiting every non-root internal node.
    When visiting node $v$:
    \begin{compactenum}
        \item Find $x \leftarrow \mrca{v}{\summaryTree^{\prime}}$
        \item If $x$ is a polytomy; $\mathcal{M}, \mathcal{N} \leftarrow \textsc{ResolvingBipart}(x, v)$. If $|\mathcal{M}| > 1$ and $\mathcal{N}\neq \emptyset$ then:
         \begin{compactenum}
            \item  $y\leftarrow \textsc{Resolve}(x, \mathcal{M}, \mathcal{N})$
            \item If $\textsc{CreatedUnsupportedNode}(y, v)$ then $\textsc{Collapse}(y)$ otherwise $\textsc{UpdateSSS}(y)$.
        \end{compactenum}
   \end{compactenum}
   \item Print $\summaryTree^{\prime}$
\end{compactenum}

\begin{algorithm} \caption{\textsc{ResolvingBipart}}\label{AlgResolvingBipart}
\begin{algorithmic}
\REQUIRE $x$ a node in a taxonomically complete tree $T$ that is $\mrca{v}{T}$
\REQUIRE $v$ a node from another tree
\ENSURE Returns a pair of sets. If both elements returned are non-empty, then 
    the pair of sets are a  bipartition of child set of $x$ and the first set 
    can be moved to a newly created daughter in order to resolve $x$ in 
    a way that is supported by $v$.
    If either set is empty, then  $v$ cannot resolve $x$.
\STATE ASSERT $\leafDes{v} \subset \leafDes{x}$
\IF{$\leafDes{x} \cap \excLeafDes{v} = \emptyset$}
    \RETURN $(\emptyset, \emptyset)$ \COMMENT{resolution might be compatible with $v$, but would not {\em supported} by $v$.}
\ENDIF
\STATE $\mathcal{M}\leftarrow \{\}$
\STATE $\mathcal{N} \leftarrow \{\}$
\FOR{each $c \in \children{x}$}
    \IF{$\leafDes{c} \cap \leafDes{v} \neq \emptyset$}
        \IF{$\leafDes{c} \cap \excLeafDes{v} \neq \emptyset$}
            \RETURN $(\emptyset, \emptyset)$ \COMMENT{can't separate $\leafDes{v}$ from $\excLeafDes{v}$ by resolving}
        \ENDIF
        $\mathcal{M}\leftarrow \mathcal{M} \cup \{ c\}$
    \ELSE
        $\mathcal{N}\leftarrow \mathcal{N} \cup \{ c\}$
    \ENDIF
\ENDFOR \\
\RETURN $(\mathcal{M}, \mathcal{N})$
\end{algorithmic}
\end{algorithm}

\begin{algorithm} \caption{\textsc{Resolve}}\label{AlgResolve}
\begin{algorithmic}
\REQUIRE $\mathcal{M}$ a set of nodes to move
\REQUIRE $\mathcal{N}$ a set of nodes to remain attached
\ENSURE Returns the new node that is a sib to the elements in $\mathcal{N}$ and the parent of the elements in $\mathcal{M}$.
\ENSURE $\texttt{desIds}$ fields of the new node is set (this field for other nodes does not change)
\ENSURE{Does \underline{not} update $\supportingPSSet{x}$ for any node $x$. This field
will be invalid until fixed!}
\STATE ASSERT all elements of $\mathcal{N}$ and $\mathcal{M}$ are sibs
\STATE $p\leftarrow \parent{x}$ for some $x\in \mathcal{N}$
\STATE create $y$ as a child of $p$.
\STATE move all elements of $\mathcal{M}$ to be children of $y$.
\STATE set $\texttt{y.desIds} = \bigcup_{z\in\mathcal{M}}\texttt{z.desIds}$
\RETURN $y$
\end{algorithmic}
\end{algorithm}

\end{document}



