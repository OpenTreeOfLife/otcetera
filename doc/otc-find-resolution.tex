\documentclass[11pt]{article}
\usepackage{amssymb}
\usepackage{amsmath}
\usepackage{pdfpages}
\usepackage{graphicx}
\usepackage[authoryear,round]{natbib}
\usepackage{algorithm,algorithmic}
\bibliographystyle{plainnat}
\usepackage{color}
\usepackage{graphicx}
\newtheorem{theorem}{Theorem}
\newtheorem{lemma}[theorem]{Lemma}
\newtheorem{corollary}[theorem]{Corollary}
\newtheorem{conjecture}[theorem]{Conjecture}
\newtheorem{definition}{Definition}
\usepackage{xspace}
\usepackage{paralist}
\usepackage{setspace}
\textwidth 6.5in
\textheight 10in
\hoffset -1in
\voffset -1.4in
\singlespacing
\parindent 0em
\parskip .4em
% Abbreviations
\newcommand{\otol}{Open Tree of Life\xspace}
\newcommand{\ps}{phylogenetic statement\xspace}
\newcommand{\pss}{phylogenetic statements\xspace}
\newcommand{\PSs}{Phylogenetic Statements\xspace}
\newcommand{\PS}{Phylogenetic Statement\xspace}
\newcommand{\SWIPSD}{Sum of Weighted Input \PSs Displayed\xspace}
\newcommand{\MSWIPSD}{Maximum \SWIPSD \xspace}
\newcommand{\newick}[1]{\texttt{#1}\xspace}
\newcommand{\otc}[0]{\texttt{otcetera}\xspace}
\newcommand{\otcprune}[0]{\texttt{otc-prune-taxonomy}\xspace}
\newcommand{\otcdecompose}[0]{\texttt{otc-uncontested-decompose}\xspace}
\newcommand{\nexson}[0]{\texttt{otNexSON}\xspace}
\newcommand{\gcmdr}[0]{\texttt{gcmdr}\xspace}
\newcommand{\simplification}[0]{\\\noindent\textsc{Simplification Action(s)}:\xspace}
\newcommand{\undoActions}[0]{\\\noindent\textsc{``Undo'' Simplification Action(s)}:\xspace}
\newcommand{\stepExplanation}[0]{\\\noindent\textsc{Explanation}:\xspace}
\newcommand{\stepInput}[0]{\\\noindent\textsc{Input}:\xspace}
\newcommand{\stepOutput}[0]{\\\noindent\textsc{Output}:\xspace}
\newcommand{\currImpl}[0]{\\\noindent\textsc{Current Impl.}:\xspace}
\newcommand{\implTODO}[0]{\\\noindent\textsc{TODO for Impl.}:\xspace}
\newcommand{\currURL}[0]{\\\noindent\textsc{URL for output}:\xspace}
\newcommand{\comment}[1]{{\color{red} \textsc{#1}}\xspace}
\newcommand{\TODO}[1]{\comment{TODO: #1}}
\newcommand{\NeedsAlgorithmicWork}{{\comment{This needs algorithmic work.}}}
\newcommand{\ProofWriteupNeeded}{{\comment{Need to write up this proof}}}

\newcommand{\incLSSS}{Include-LeafSet support statement\xspace}
\newcommand{\incLSSSs}{Include-LeafSet support statements\xspace}

% Notation
\newcommand{\pssInOptimalTree}{\ensuremath{\hat{\mathcal{G}}}\xspace}
\newcommand{\pssFrom}[1]{\ensuremath{\mathcal{G}(#1)}\xspace}
\newcommand{\tripleSetInOptimal}{\ensuremath{\hat{\mathcal{R}}}\xspace}
\newcommand{\leafLabels}[1]{\ensuremath{\mathcal{L}(#1)}}
\newcommand{\parent}[1]{\mbox{parent}(#1)}
\newcommand{\children}[1]{\mbox{children}(#1)}
\newcommand{\nodes}[1]{\mbox{nodes}(#1)}
\newcommand{\treeRoot}[1]{\mbox{root}(#1)}
\newcommand{\taxonomy}[0]{\ensuremath{\mathbb{T}}\xspace}
\newcommand{\prunedTaxonomy}[0]{\ensuremath{\mathbb{T}_P}\xspace}
\newcommand{\phyloInputs}[0]{\ensuremath{\mathcal{T}}}
\newcommand{\expandedPhylo}[0]{\ensuremath{\mathcal{T}_{E}}\xspace}
\newcommand{\prunedSummary}[0]{\ensuremath{\mathcal{S}_{P}}\xspace}
\newcommand{\summaryTree}[0]{\ensuremath{\mathcal{S}}\xspace}
% verbatim, verbatim notation for a \ps
\newcommand{\vvps}[2]{\ensuremath{{#1}\downarrow{#2}}}
% leaf set, verbatim notation for a \ps
\newcommand{\lvps}[2]{\ensuremath{\leafLabels{#1}\downarrow{#2}}}
\newcommand{\leafComp}[2]{\ensuremath{\widetilde{\mathcal{L}_{#2}}\left({#1}\right)}}
\newcommand{\displaysPred}[2]{\ensuremath{\mathbb{I}_d(#1, #2)}}

\newcommand{\leafDes}[1]{\ensuremath{\mathcal{L}_d(#1)}}
\newcommand{\excLeafDes}[1]{\ensuremath{\mathcal{L}_e(#1)}}
%\newcommand{\children}[1]{\ensuremath{\mathcal{C}(#1)}}

\newcommand{\mrca}[2]{\ensuremath{\mbox{mrca}(#1, #2)}}
\newcommand{\treeOf}[1]{\ensuremath{\mbox{tree}(#1)}}
\newcommand{\leafSet}[1]{\leafLabels{#1}}}
\newcommand{\cLeafSet}[1]{\texttt{{#1}.leafLSet}}}
\newcommand{\cDes}[1]{\texttt{{#1}.desIds}}}

\newcommand{\supportingPSSetFull}[2]{\ensuremath{\mbox{\texttt{SuppStatementSet}}[#1, #2]}}
\newcommand{\supportingPSSet}[1]{\ensuremath{\mbox{\texttt{SSS}}[#1]}}
\DeclareMathOperator*{\argmax}{\arg\!\max}

\usepackage{xr}
\externaldocument[S-]{summarizing-taxonomy-plus-trees}
\newcommand{\summDoc}{\href{http://phylo.bio.ku.edu/ot/summarizing-taxonomy-plus-trees.pdf}{the otcetera summary}\xspace}
\newcommand{\summRef}[1]{Section \ref{S-#1} of \summDoc}
\usepackage{hyperref}
\hypersetup{backref,  linkcolor=blue, citecolor=black, colorlinks=true, hyperindex=true}
\begin{document}
\section*{\texttt{otc-find-resolution}}
This document describes the behavior of \texttt{otc-find-resolution} when the \texttt{-u}
    option is used to request resolution of tree without introducing unsupported nodes.

\subsection*{Background and notation}
Consider a supertree \summaryTree that displays a set of input \pss \pssInOptimalTree.
We say that \summaryTree is {\em minimal} with respect \pssInOptimalTree, if there is
    no edge in \summaryTree that can be collapse to result in a smaller tree $\summaryTree^{\prime}$
    which still displays all of the \pss in \pssInOptimalTree.

$\leafLabels{T}$ is the set of leaf labels associated with the tips of a tree.
Let $\leafDes{n}$ represent the set of leaf label that are associated with node that are descendants
    of node $n$.
$\excLeafDes{n, T}$ is the set of leaf labels of the tree T which are {\em not} descendants of $n$.
Thus $\leafLabels{T} = \leafDes{n} \cup \excLeafDes(n, T)$ and $\leafDes{n} \cap \excLeafDes(n, T)=\emptyset$ for any node $n$ in $T$.

$\pssFrom{T}$ is the set of \pss that correspond to the non-root, clusters of tree $T$.
If $T$ denotes a set or series of trees, then this notation is used to indicate the union of the \pss from each tree.

$\children{n}$ is the set of nodes that are children of $n$.

\subsection*{Input}
\begin{compactitem}
    \item A tree representing the full taxonomy, \taxonomy
    \item A summary tree, $\summaryTree$, which has no nodes of outdgree-1 and which is minimal wrt \pssInOptimalTree.
    \item A series of input phylogenetic estimates  $[T_1, T_2, \ldots T_{N-1}]$ in ranked order of importance (1 being the most important)
\end{compactitem}
The full set of inputs is taken to be: $\mathcal{T} = [T_1, T_2, \ldots T_{N}]$ where $T_N = \taxonomy$.
    $\pssInOptimalTree$ is the subset of $\pssFrom{\mathcal{T}}$ which are displayed by $\summaryTree$.

\subsection*{Goal}
Print a tree $\summaryTree^{\prime}$ which:
\begin{compactitem}
    \item is a more resolved form of $\summaryTree$.
    \item is minimal wrt $\pssInOptimalTree^{\prime}$ where $\pssInOptimalTree \subset \pssInOptimalTree^{\prime}$ and $\pssInOptimalTree^{\prime} \subset \pssFrom{\mathcal{T}}$. and 
    \item groupings are added to the tree in the order of the ranking of the trees (to maximize the \SWIPSD score)
\end{compactitem}

The output should be the highest scoring resolution of $\summaryTree$ under the \SWIPSD score when the
    weights of for each tree is $>>$ than the weight of the next ranked tree.

\section*{Implementation}
\begin{compactenum}
    \item All trees are read.
    \item Any tip $n$ in $\mathcal{T}$ mapped to a non-terminal taxonis expanded as described in \summRef{expandNonTermPar}. This creates of $\leafDes{n}$ attached to $\parent{n}$ and removes $n$ from the tree.
\end{compactenum}
\end{document}